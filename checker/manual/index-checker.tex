%\htmlhr
\chapter{Index Checker for array bounds\label{index-checker}}

The Index Checker warns about unsafe array
accesses.
It protects against Java's
ArrayIndexOutOfBoundsException.

To do so, it keeps track of the ranges of values potential
indexes may take on.  The Index Checker is actually a combination of
two different checkers: the Lower
Bound Checker and the Upper Bound Checker.  The Lower Bound Checker
will issue warning when the variable being used as an index may
potentially be too low; the Upper Bound Checker warns about accesses
that may be too high. Both checkers use automatic introduction and
transfer rules which should make it unnecessary for the programmer to
write annotations, so long as \textit{defensive programming} is being
practiced---that is, so long as the programmer is careful to check
that indices are within bounds before using them, the Index Checker
requires no annotations from the programmer.

\section{Running the Index Checker\label{index-running}}

To run the Index Checker, run the command

\begin{Verbatim}
javac -processor index MyJavaFile.java
\end{Verbatim}


\section{The Lower Bound Checker\label{index-lowerbound}}

The Lower Bound checker looks for accesses into arrays that might be
too low.  It warns its users about any such possible accesses. The
Lower Bound checker's type system is linear and fairly
straightforward. It primarily keeps track of each integer's relation
to zero, using the following qualifiers (all of which are inferred
automatically):

\begin{description}
\item[\refqualclass{checker/lowerbound/qual}{Positive}]
  This variable has value 1 or greater.
  Variables of this type are safe for use as array indices.
\item[\refqualclass{checker/lowerbound/qual}{NonNegative}]
  This variable has value 0 or greater.
  Variables of this type are safe for use as array indices.
\item[\refqualclass{checker/lowerbound/qual}{GreaterThanOrEqualToNegativeOne}]
  This variable has value -1 or greater.
  Variables of this type are unsafe for use as array indices.
\item[\refqualclass{checker/lowerbound/qual}{LowerBoundUnknown}]
  There is no information about the value of this variable.
  This type is the top type.
\end{description}

\begin{figure}
\includeimage{lowerbound}{7cm}
\caption{The type hierarchy used by the Lower Bound Checker.}
\label{fig-lowerbound-types}
\end{figure}

Users rarely have to write these annotations.  They are inferred on
constants, and at each program operation, the Lower Bound
Checker uses the rules of arithmetic to determine the type for the result.
The full list of rules appears in file \<lowerbound\_rules.txt> in the
source code's \<lowerbound/> directory.


\section{The Upper Bound Checker\label{index-upperbound}}

The Upper Bound Checker warns about array accesses whose index might be
too high. The Upper Bound Checker uses the following
qualifiers:

\begin{description}
\item[\refqualclasswithparams{checker/upperbound/qual}{LessThanLength}{String[] names}]
  A variable with this type
  has value less than the length of each array listed in names.
\item[\refqualclasswithparams{checker/upperbound/qual}{EqualToLength}{String[] names}]
  A variable with this type
  has value equal to the length of each array listed in names.
\item[\refqualclasswithparams{checker/upperbound/qual}{LessThanOrEqualToLength}{String[] names}]
  A variable with this type
  has value less than or equal to the length of each array listed in names.
\item[\refqualclass{checker/upperbound/qual}{UpperBoundUnknown}]
  There is no information about the value of this variable.
  This type is the top type.
\end{description}
