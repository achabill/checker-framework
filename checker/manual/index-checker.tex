%\htmlhr
\chapter{Index Checker for array bounds\label{index-checker}}

The Index Checker warns about unsafe array accesses.  It protects
against Java's
\sunjavadoc{java/lang/ArrayIndexOutOfBoundsException.html}{ArrayIndexOutOfBoundsException}. A
class checked by the
Index Checker  will not throw an ArrayIndexOutOfBoundException unless the
index has overflowed or underflowed. The Index Checker maintains information
for each sequence-like data structure and will report unsafe accesses to each.
It is designed primarily for arrays and Lists.

The Index Checker maintains an estimate of the range of values for each
index expression.  There are two ways an expression can fail to be safe for use as an
index: either it can be too low or it can be too high. The Index
Checker will warn about both types of dangerous accesses, and will
issue different warnings for the two.

Typically, programmers do not need to write annotations to use the
Index Checker. The Index Checker will infer properties of indexes from
the code around them.  For example, it will infer that \<x> is positive
within the then block of an \code{if (x > 0)} statement.


\section{Running the Index Checker\label{index-running}}

To run the Index Checker, run the command

\begin{alltt}
  javac -processor index \emph{MyJavaFile}.java
\end{alltt}

Internally, the Index Checker is composed of three checkers: the Lower
Bound Checker (section~\ref{index-lowerbound}), which checks whether potential accesses may be too low;
the Upper Bound Checker (section~\ref{index-upperbound}), which checks whether potential accesses
may be too high; and the MinLen Checker (section~\ref{index-minlen}), which estimates the minimum
length of each sequence.


\section{The Lower Bound Checker\label{index-lowerbound}}

The Lower Bound Checker warns about indices that might be
too low to access a zero-indexed sequence. The
Lower Bound Checker's type system uses the following qualifiers.

\begin{description}
\item[\refqualclass{checker/lowerbound/qual}{Positive}]
  The value is 1 or greater, so it is not too low to be used as an index.
\item[\refqualclass{checker/lowerbound/qual}{NonNegative}]
  The value is 0 or greater, so it is not too low to be used as an index.
\item[\refqualclass{checker/lowerbound/qual}{GTENegativeOne}]
  The value is -1 or greater.
  It may not be used as an index for a sequence, because it might be too low.
\item[\refqualclass{checker/lowerbound/qual}{LowerBoundUnknown}]
  There is no information about the value.
  It may not be used as an index for a sequence, because it might be too low.
\end{description}

\begin{figure}
  \includeimage{lowerbound}{7cm}
  \caption{The type hierarchy used by the Lower Bound Checker.}
  \label{fig-lowerbound-types}
\end{figure}

%% True, but implementation details don't seem relevant to someone reading the manual.
% Internally, the Lower Bound Checker uses both the Constant Value Checker and the MinLen
% Checker. The Constant Value Checker is used to reason about constants, and to assign
% them the appropriate Lower Bound types. The MinLen Checker is used to give correct types
% to calls to \code{arr.length}, where \code{arr} is any array. The Lower Bound checker
% implements a large number of transfer rules. Because there are so many, they are
% not documented here; the code implementing them contains JavaDoc comments that describe
% the rules it implements.

Typically, a programmer only needs to write these annotations on
fields and method signatures, because the Lower Bound Checker infers
qualifiers within method bodies.
Detailed documentation about the inferences performed can be found
in the Javadoc comments in its source code.


\section{The Upper Bound Checker\label{index-upperbound}}

The Upper Bound Checker warns about accesses to sequences with indexes that might be
too high.
The Upper Bound Checker's type system uses the following qualifiers.
Typically, a programmer only needs to write these annotations on
fields and method signatures, because the Upper Bound Checker infers
qualifiers within method bodies.

\begin{description}
\item[\refqualclasswithparams{checker/upperbound/qual}{LTLengthOf}{String[] names}]
  An expression with this type
  has value less than the length of each sequence listed in \<names>.
  The expression may be used as an index into any of those sequences.
  \code{@LTLengthOf({"a", "b"})} is a subtype of both
  \code{@LTLengthOf("a")} and \code{@LTLengthOf("b")}.
\item[\refqualclasswithparams{checker/upperbound/qual}{LTEqLengthOf}{String[] names}]
  An expression with this type
  has value less than or equal to the length of each sequence listed in \<names>.
  It may not be used as an index for a sequence, because it might be too high.
  \code{@LTEqLengthOf({"a", "b"})} is a subtype of both
  \code{@LTEqLengthOf("a")} and \code{@LTEqLengthOf("b")}.
\item[\refqualclass{checker/upperbound/qual}{UpperBoundUnknown}]
  There is no information about the value of an expression with this type.
  It may not be used as an index for a sequence, because it might be too high.
  This type is the top type, and should never need to be written by the
  programmer.
\item[\refqualclass{checker/upperbound/qual}{UpperBoundBottom}]
  This is the bottom type for the Upper Bound type system. It should
  never need to be written by the programmer.
  \end{description}

\begin{figure}
  \includeimage{upperbound}{7cm}
  \caption{The type hierarchy used by the Upper Bound Checker.}
  \label{fig-upperbound-types}
\end{figure}

% Like the Lower Bound checker, the Upper Bound checker relies on both the Constant
% Value Checker and the MinLen checker. It uses the Constant Value Checker to determine
% precisely the value of particular indexes, and uses the MinLen checker both in the way
% described above (for constant indexes) and to determine when potential indexes should
% be assigned
% \refqualclasswithparams{checker/upperbound/qual}{LTLengthOf}{String[] names}] or
% \refqualclasswithparams{checker/upperbound/qual}{LTEqLengthOf}{String[] names}].


\section{The MinLen Checker\label{index-minlen}}

The MinLen Checker estimates, for each sequence expression, how long its value
might be at run time.  The MinLen Checker computes a minimum length that
the sequence is guaranteed to have.

The main reason for the MinLen Checker is to handle fixed-length arrays
that are indexed by a compile-time constant.
There are two different situations when \<arr[i]> is legal because \<i> is
not too large:
\begin{itemize}
\item
  when \<i> is known to be less than the length of \<arr> independent of the length of
  \<arr>.  This fact is represented by giving \<i> the type \refqualclasswithparams{checker/upperbound/qual}{LTLengthOf}{"arr"}.
\item
  when \<i> is a compile-time constant but the length of \<arr> is known to
  be greater.  For example, consider the following code:
\end{itemize}

\begin{Verbatim}
  String getSecondElement(String[] arr) {
    return arr[1];
  }
\end{Verbatim}
  This is legal if \<arr> has at least two elements, which can be indicated
  in this way:
\begin{Verbatim}
  String getSecondElement(String @MinLen(2) [] arr) {
    return arr[1];
  }
\end{Verbatim}

The MinLen Checker uses the following type hierarchy:

\begin{description}
\item[\refqualclasswithparams{checker/minlen/qual}{MinLen}{int value}]
  An expression with this type represents a sequential structure
  with at least \code{value} elements.  The default annotation is
  \code{@MinLen(0)}, since an arbitrary object has at least 0 elements.
  In general, \code{@MinLen(x)} is a subtype of \code{@MinLen(x-1)}.
\item[\refqualclass{checker/minlen/qual}{MinLenBottom}]
  This is the bottom type for the MinLen type system.
  Programmers should rarely need to write it.
  \code{null} has this type.
  \end{description}

\begin{figure}
  \includeimage{minlen}{7cm}
  \caption{The type hierarchy used by the MinLen Checker.}
  \label{fig-minlen-types}
\end{figure}
