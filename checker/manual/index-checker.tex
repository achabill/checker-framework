%\htmlhr
\chapter{Index Checker for array bounds\label{index-checker}}

This checker warns about the potential presence of unsafe array
accesses.  To do so, it keeps track of the ranges of values potential
indexes may take on.  The Index Checker is actually a combination of
two different checkers, each of which is run independently: the Lower
Bound Checker and the Upper Bound Checker.  The Lower Bound Checker
will issue warning when the variable being used as an index may
potentially be too low; the Upper Bound Checker warns about accesses
that may be too high. Both checkers use automatic introduction and
transfer rules which should make it unnecessary for the programmer to
write annotations, so long as \textit{defensive programming} is being
practiced---that is, so long as the programmer is careful to check
that indices are within bounds before using them, the Index Checker
requires no annotations from the programmer.

\section{Running the Index Checker\label{index-running}}

To run the Index Checker, run the command

\begin{Verbatim}
javac -processor IndexChecker <JavaFileName>.java
\end{Verbatim}

The Lower Bound and Upper Bound checkers can also be run
independently; simply substitute ``LowerBoundChecker'' or
``UpperBoundChecker'' for ``IndexChecker'' in the above command as
desired.

\section{The Lower Bound Checker\label{index-lowerbound}}

The Lower Bound checker looks for accesses into arrays that might be
too low.  It warns its users about any such possible accesses. The
Lower Bound checker's type system is linear and fairly
straightforward. It primarily keeps track of each integer's relation
to zero, using the following qualifiers (all of which are inferred
automatically):

\begin{description}
\item[\refqualclass{checker/lowerbound/qual}{LowerBoundUnknown}]
  There is no information about the value of this variable.
  This type is the top type. Variables of this type are
  unsafe for use as array indices.
\item[\refqualclass{checker/lowerbound/qual}{GreaterThanOrEqualToNegativeOne}]
  This variable has value -1 or greater. This type is a subtype of LowerBoundUnknown.
  Variables of this type are unsafe for use as array indices.
\item[\refqualclass{checker/lowerbound/qual}{NonNegative}]
  This variable has value 0 or greater. This type is a subtype of GreaterThanOrEqualToNegativeOne.
  Variables of this type are safe for use as array indices.
\item[\refqualclass{checker/lowerbound/qual}{Positive}]
  This variable has value 1 or greater. This type is a subtype of NonNegative.
  Variables of this type are safe for use as array indices.
\end{description}

\begin{figure}
\includeimage{lowerbound}{7cm}
\caption{The type hierarchy used by the Lower Bound Checker.}
\label{fig-lowerbound-types}
\end{figure}

The transfer rules for this type hierarchy are intuitive. For brevity,
the full list of rules is not reproduced here; it can be found in the file
``lowerbound\_rules.txt'' in the source code's ``lowerbound'' directory.
The essence of the rules is that of basic arithmetic: adding one to a
NonNegative turns it into a Positive, checking if an Unknown is greater
than zero makes it Positive in the then branch, etc.
