\htmlhr
\chapter{Nullness Checker\label{nullness-checker}}

If the Nullness Checker issues no warnings for a given program, then
running that program will never throw a null pointer exception.  This
guarantee enables a programmer to prevent errors from occurring when a
program is run.  See Section~\ref{nullness-checks} for more details about
the guarantee and what is checked.

The most important annotations supported by the Nullness Checker are
\refqualclass{checker/nullness/qual}{NonNull} and
\refqualclass{checker/nullness/qual}{Nullable}.
\refqualclass{checker/nullness/qual}{NonNull} is rarely written, because it is
the default.  All of the annotations are explained in
Section~\ref{nullness-annotations}.

To run the Nullness Checker, supply the
\code{-processor org.checkerframework.checker.nullness.NullnessChecker}
command-line option to javac.  For
examples, see Section~\ref{nullness-example}.

The NullnessChecker is actually an ensemble of three pluggable
type-checkers that work together: the Nullness Checker proper (which is the
main focus of this chapter), the Initialization Checker
(Section~\ref{initialization-checker}), and the Map Key Checker
(\chapterpageref{map-key-checker}).
Their type hierarchies are completely independent, but they work together
to provide precise nullness checking.


\section{What the Nullness Checker checks\label{nullness-checks}}

The checker issues a warning in these cases:

\begin{enumerate}

\item
  When an expression of non-\refqualclass{checker/nullness/qual}{NonNull} type
  is dereferenced, because it might cause a null pointer exception.
  Dereferences occur not only when a field is accessed, but when an array
  is indexed, an exception is thrown, a lock is taken in a synchronized
  block, and more.  For a complete description of all checks performed by
  the Nullness Checker, see the Javadoc for
  \refclass{checker/nullness}{NullnessVisitor}.

\item
  When an expression of \refqualclass{checker/nullness/qual}{NonNull} type
  might become null, because it
  is a misuse of the type:  the null value could flow to a dereference that
  the checker does not warn about.

  As a special case of an of \refqualclass{checker/nullness/qual}{NonNull}
  type becoming null, the checker also warns whenever a field of
  \refqualclass{checker/nullness/qual}{NonNull} type is not initialized in a
  constructor.  Also see the discussion of the \code{-Alint=uninitialized}
  command-line option below.

\end{enumerate}

This example illustrates the programming errors that the checker detects:

\begin{Verbatim}
  @Nullable Object   obj;  // might be null
  @NonNull  Object nnobj;  // never null
  ...
  obj.toString()         // checker warning:  dereference might cause null pointer exception
  nnobj = obj;           // checker warning:  nnobj may become null
  if (nnobj == null)     // checker warning:  redundant test
\end{Verbatim}

Parameter passing and return values are checked analogously to assignments.

The Nullness Checker also checks the correctness, and correct use, of
rawness annotations for checking initialization (see
Section~\ref{initialization-rawness-checker}) and of map key annotations (see
\chapterpageref{map-key-checker}).


The checker performs additional checks if certain \code{-Alint}
command-line options are provided.  (See
Section~\ref{alint} for more details about the \code{-Alint}
command-line option.)

\label{lint-nulltest-section}\label{lint-uninitialized-section}
\begin{enumerate}
\item
  Options that control soundness:

\begin{itemize}
\item
  \label{lint-nonnullarraycomponents-item}%
  If you supply the \code{-Alint=forbidnonnullarraycomponents} command-line
  option, then the checker warns if it encounters an array creation
  with a non-null component type.
  See Section~\ref{nullness-arrays} for a discussion.
  % TODO: this option is temporary and the sound option
  % should become the default.
\end{itemize}

\item
  Options that warn about poor code style:

\begin{itemize}
\item
  \label{lint-nulltest-item}%
  If you supply the \code{-Alint=redundantNullComparison} command-line option, then the
  checker warns when a null check is performed against a value that is
  guaranteed to be non-null, as in \code{("m" == null)}.  Such a check is
  unnecessary and might indicate a programmer error or misunderstanding.
  The lint option is disabled by default because sometimes such checks are
  part of ordinary defensive programming.

\item
  \label{lint-uninitialized-item}%
  If you supply the \code{-Alint=uninitialized} command-line option, then
  the checker warns if a constructor fails to initialize any field,
  including \refqualclass{checker/nullness/qual}{Nullable} types and primitive
  types.  Such a warning is unrelated to whether your code might throw a
  null pointer exception.  However, you might want to enable this warning
  because it is better code style to supply an explicit initializer, even
  if there is a default value such as \code{0} or \code{false}.
  This command-line option does not affect the Nullness Checker's tests
  that fields of \refqualclass{checker/nullness/qual}{NonNull} type are
  initialized --- such initialization is mandatory, not optional.
\end{itemize}

\end{enumerate}


\section{Nullness annotations\label{nullness-annotations}}

The Nullness Checker uses three separate type hierarchies:  one for nullness,
one for rawness (Section~\ref{initialization-rawness-checker}),
and one for map keys (\chapterpageref{map-key-checker})
The Nullness Checker has four varieties of annotations:  nullness
type qualifiers, nullness method annotations, rawness type qualifiers, and
map key type
qualifiers.

\subsection{Nullness qualifiers\label{nullness-qualifiers}}

The nullness hierarchy contains these qualifiers:

\begin{description}

\item[\refqualclass{checker/nullness/qual}{Nullable}]
  indicates a type that includes the null value.  For example, the type \code{Boolean}
  is nullable:  a variable of type \code{Boolean} always has one of the
  values \code{TRUE}, \code{FALSE}, or \code{null}.

\item[\refqualclass{checker/nullness/qual}{NonNull}]
  indicates a type that does not include the null value.  The type
  \code{boolean} is non-null; a variable of type \code{boolean} always has
  one of the values \code{true} or \code{false}.  The type \code{@NonNull
    Boolean} is also non-null:  a variable of type \code{@NonNull Boolean}
  always has one of the values \code{TRUE} or \code{FALSE} --- never
  \code{null}.  Dereferencing an expression of non-null type can never cause
  a null pointer exception.

  The \<@NonNull> annotation is rarely written in a program, because it is
  the default (see Section~\ref{null-defaults}).

\item[\refqualclass{checker/nullness/qual}{PolyNull}]
  indicates qualifier polymorphism.  For a description of
  \refqualclass{checker/nullness/qual}{PolyNull}, see
  Section~\ref{qualifier-polymorphism}.

\item[\refqualclass{checker/nullness/qual}{MonotonicNonNull}]
  indicates a reference that may be \code{null}, but if it ever becomes
  non-\code{null}, then it never becomes \code{null} again.  This is
  appropriate for lazily-initialized fields, among other uses.  When the
  variable is read, its type is treated as
  \refqualclass{checker/nullness/qual}{Nullable}, but when the variable is
  assigned, its type is treated as
  \refqualclass{checker/nullness/qual}{NonNull}.

  \begin{sloppypar}
  Because the Nullness Checker works intraprocedurally (it analyzes one
  method at a time), when a \code{MonotonicNonNull} field is first read within a
  method, the field cannot be assumed to be non-null.  The benefit of
  MonotonicNonNull over Nullable is its different interaction with
  flow-sensitive type qualifier refinement (Section~\ref{type-refinement}).
  After a check of a MonotonicNonNull
  field, all subsequent accesses \emph{within that method} can be assumed
  to be NonNull, even after arbitrary external method calls that have
  access to the given field.
  \end{sloppypar}

  It is permitted to initialize a MonotonicNonNull field to null, but the
  field may not be assigned to null anywhere else in the program.  If you
  supply the \<noInitForMonotonicNonNull> lint flag (for example, supply
  \<-Alint=noInitForMonotonicNonNull> on the command line), then
  @MonotonicNonNull fields are not allowed to have initializers.

  Use of \<@MonotonicNonNull> on a static field is a code smell:  it may
  indicate poor design.  You should consider whether it is possible to make
  the field a member field that is set in the constructor.

\end{description}

Figure~\ref{fig-nullness-hierarchy} shows part of the type hierarchy for the
Nullness type system.
(The annotations exist only at compile time; at run time, Java has no
multiple inheritance.)

\begin{figure}
\includeimage{nullness}{3.5cm}
\caption{Partial type hierarchy for the Nullness type system.
Java's \<Object> is expressed as \<@Nullable Object>.  Programmers can omit
most type qualifiers, because the default annotation
(Section~\ref{null-defaults}) is usually correct.
The Nullness Checker verifies three type hierarchies:  this one for
nullness, one for initialization (Section~\ref{initialization-checker}),
and one for map keys (\chapterpageref{map-key-checker}).}
\label{fig-nullness-hierarchy}
\end{figure}


\subsection{Nullness method annotations\label{nullness-method-annotations}}

The Nullness Checker supports several annotations that specify method
behavior.  These are declaration annotations, not type annotations:  they
apply to the method itself rather than to some particular type.

\begin{description}

\item[\refqualclass{checker/nullness/qual}{RequiresNonNull}]
  indicates a method precondition:  The annotated method expects the
  specified variables (typically field references) to be non-null when the
  method is invoked.

\item[\refqualclass{checker/nullness/qual}{EnsuresNonNull}]
\item[\refqualclass{checker/nullness/qual}{EnsuresNonNullIf}]
  indicates a method postcondition.  With \<@EnsuresNonNull>, the given
  expressions are non-null after the method returns; this is useful for a
  method that initializes a field, for example.  With
  \<@EnsuresNonNullIf>, if the annotated
  method returns the given boolean value (true or false), then the given
  expressions are non-null.  See Section~\ref{conditional-nullness} and the
  Javadoc for examples of their use.

\end{description}


\subsection{Initialization qualifiers\label{initialization-qualifiers-overview}}

The Nullness Checker invokes an Initialization Checker, whose annotations indicate whether
an object is fully initialized --- that is, whether all of its fields have
been assigned.

\begin{description}
\item[\refqualclass{checker/initialization/qual}{Initialized}]
\item[\refqualclass{checker/initialization/qual}{UnknownInitialization}]
\item[\refqualclass{checker/initialization/qual}{UnderInitialization}]
\end{description}

\noindent
Use of these annotations can help you to type-check more
code.  Figure~\ref{fig-initialization-hierarchy} shows its type hierarchy.  For
details, see Section~\ref{initialization-checker}.

A slightly simpler variant, called the Rawness Initialization Checker, is also available:

\begin{description}
\item[\refqualclass{checker/nullness/qual}{Raw}]
\item[\refqualclass{checker/nullness/qual}{NonRaw}]
\item[\refqualclass{checker/nullness/qual}{PolyRaw}]
\end{description}

\noindent
Figure~\ref{fig-rawness-hierarchy} shows its type hierarchy.  For
details, see Section~\ref{initialization-rawness-checker}.


\subsection{Map key qualifiers\label{map-key-qualifiers}}

\begin{description}
\item[\refqualclass{checker/nullness/qual}{KeyFor}]
\end{description}
    indicates that a value is a key for a given map --- that is, indicates
    whether \code{map.containsKey(value)} would evaluate to \code{true}.

This annotation is checked by a Map Key Checker
(\chapterpageref{map-key-checker}) that the Nullness Checker
invokes.  The \refqualclass{checker/nullness/qual}{KeyFor} annotation enables
the Nullness Checker to treat calls to
\sunjavadoc{java/util/Map.html\#get-java.lang.Object-}{\code{Map.get}}
precisely rather than assuming it may always return null.  In particular,
a call \<mymap.get(mykey)> returns a non-\<null> value if two conditions
are satisfied:
\begin{enumerate}
\item \<mymap>'s values are all non-\<null>; that is, \<mymap> was
  declared as \code{Map<\emph{KeyType}, @NonNull \emph{ValueType}>}.  Note
  that \<@NonNull> is the default type, so it need not be written explicitly.
\item \<mykey> is a key in \<mymap>; that is, \<mymap.containsKey(mykey)>
  returns \<true>.  You express this fact to the Nullness Checker by
  declaring \<mykey> as \<@KeyFor("mymap") \emph{KeyType} mykey>.  For a
  local variable, you generally do not need to write the
  \<@KeyFor("mymap")> type qualifier, because it can be inferred.
\end{enumerate}
\noindent
If either of these two conditions is violated, then \<mymap.get(mykey)> has
the possibility of returning \<null>.

\section{Writing nullness annotations\label{writing-nullness-annotations}}

\subsection{Implicit qualifiers\label{nullness-implicit-qualifiers}}

The Nullness Checker
adds implicit qualifiers, reducing the number of annotations that must
appear in your code (see Section~\ref{effective-qualifier}).
For example, enum types are implicitly non-null, so you never need to write
\<@NonNull MyEnumType>.

For more information about implicitly-added nullness qualifiers, see the
implementation of \refclass{checker/nullness}{NullnessAnnotatedTypeFactory}.



\subsection{Default annotation\label{null-defaults}}

Unannotated references are treated as if they had a default annotation.
The standard defaulting rule is CLIMB-to-top,  described in
Section~\ref{climb-to-top}.  Its effect is to default all types to
\<@NonNull>, except that \<@Nullable> is used for casts, locals,
instanceof, and implicit bounds.  A user can choose a different defaulting
rule by writing a \refqualclass{framework/qual}{DefaultQualifier} annotation on a
 package, class, or method.  In the example below, fields are defaulted to
\<@Nullable> instead of \<@NonNull>.

\begin{Verbatim}
@DefaultQualifier(value = Nullable.class, locations = TypeUseLocation.FIELD)
class MyClass {
    Object nullableField = null;
    @NonNull Object nonNullField = new Object();
}
\end{Verbatim}


%% Cut this to shorten the section.  Most users won't care about it.
% %BEGIN LATEX
% \begin{sloppy}
% %END LATEX
% Here are three possible default rules you may wish to use.  Other rules are
% possible but are not as useful.
% \begin{itemize}
% \item
%   \refqualclass{checker/nullness/qual}{Nullable}:  Unannotated types are regarded as possibly-null, or
%   nullable.  This default is backward-compatible with Java, which permits
%   any reference to be null.  You can activate this default by writing
%   a \code{@DefaultQualifier(Nullable.class)} annotation on a
%   % package/
%   class or method
%   % /variable
%   declaration.
% \item
%   \refqualclass{checker/nullness/qual}{NonNull}:  Unannotated types are treated as non-null.
%   % This may leads to fewer annotations written in your code.
%   You can activate this
%   default via the
%   \code{@DefaultQualifier(NonNull.class)} annotation.
% \item
%   Non-null except locals (NNEL):  Unannotated types are treated as
%   \refqualclass{checker/nullness/qual}{NonNull}, \emph{except} that the
%   unannotated raw type of a local variable is treated as
%   \refqualclass{checker/nullness/qual}{Nullable}.  (Any generic arguments to a
%   local variable still default to
%   \refqualclass{checker/nullness/qual}{NonNull}.)  This is the standard
%   behavior.  You can explicitly activate this default via the
%   \code{@DefaultQualifier(value=NonNull.class,
%     locations=\discretionary{}{}{}\{DefaultLocation\discretionary{}{}{}.ALL\_EXCEPT\_LOCALS\})}
%   annotation.
%
%   The NNEL default leads to the smallest number of explicit annotations in
%   your code~\cite{PapiACPE2008}.  It is what we recommend.  If you do not
%   explicitly specify a different default, then NNEL is the default.
% \end{itemize}
% %BEGIN LATEX
% \end{sloppy}
% %END LATEX


\subsection{Conditional nullness\label{conditional-nullness}}

The Nullness Checker supports a form of conditional nullness types, via the
\refqualclass{checker/nullness/qual}{EnsuresNonNullIf} method annotations.
The annotation on a method declares that some expressions are non-null, if
the method returns true (false, respectively).

Consider \sunjavadoc{java/lang/Class.html}{java.lang.Class}.
Method
\sunjavadoc{java/lang/Class.html\#getComponentType--}{Class.getComponentType()}
may return null, but it is specified to return a non-null value if
\sunjavadoc{java/lang/Class.html\#isArray--}{Class.isArray()} is
true.
You could declare this relationship in the following way (this particular
example is already
done for you in the annotated JDK that comes with the Checker Framework):

\begin{Verbatim}
  class Class<T> {
    @EnsuresNonNullIf(expression="getComponentType()", result=true)
    public native boolean isArray();

    public native @Nullable Class<?> getComponentType();
  }
\end{Verbatim}

A client that checks that a \code{Class} reference is indeed that of an array,
can then de-reference the result of \code{Class.getComponentType} safely
without any nullness check.  The Checker Framework source code itself
uses such a pattern:

\begin{Verbatim}
    if (clazz.isArray()) {
      // no possible null dereference on the following line
      TypeMirror componentType = typeFromClass(clazz.getComponentType());
      ...
    }
\end{Verbatim}

Another example is \sunjavadoc{java/util/Queue.html\#peek--}{Queue.peek}
and \sunjavadoc{java/util/Queue.html\#poll--}{Queue.poll}, which return
non-null if \sunjavadoc{java/util/Collection.html\#isEmpty--}{isEmpty}
returns false.

The argument to \code{@EnsuresNonNullIf} is a Java expression, including method calls
(as shown above), method formal parameters, fields, etc.; for details, see
Section~\ref{java-expressions-as-arguments}.
More examples of the use of these annotations appear in the Javadoc for
\refqualclass{checker/nullness/qual}{EnsuresNonNullIf}.


\subsection{Nullness and arrays\label{nullness-arrays}}

Suppose that you declare an array to contain non-null elements.
Currently, the Nullness Checker does not verify that all the elements of the
array are initialized to a non-null value.  This is an unsoundness in the
checker.

To make the Nullness Checker conservatively reject code that may leave a
non-null value in an array, use the command-line option
\code{-Alint=forbidnonnullarraycomponents}.  The option is currently disabled
because it makes the checker issue many false positive errors.
% TODO: flip the default after collecting more data.

When the \code{-Alint=forbidnonnullarraycomponents} option is supplied,
the following is not allowed:

\begin{Verbatim}
  Object [] oa = new Object[10]; // error
\end{Verbatim}

\noindent
(recall that \<Object> means the same thing as \<@NonNull Object>).

Instead, your code needs to create a nullable or lazy-nonnull array, initialize
each component, and then assign the result to a non-null array:

\begin{Verbatim}
  @MonotonicNonNull Object [] temp = new @MonotonicNonNull Object[10];
  for (int i = 0; i < temp.length; ++i) {
    temp[i] = new Object();
  }
  @SuppressWarnings("nullness") // temp array is now fully initialized
  @NonNull Object [] oa = temp;
\end{Verbatim}

Note that the checker is currently not powerful enough to ensure that
each array component was initialized. Therefore, the last assignment
needs to be trusted:  that is, a programmer must verify that it is safe,
then write a \<@SuppressWarnings> annotation.

% TODO: explain more aspects, give more examples.



\subsection{Run-time checks for nullness\label{nullness-runtime-checks}}

When you perform a run-time check for nullness, such as \<if (x != null)
...>, then the Nullness Checker refines the type of \<x> to
\<@NonNull>.  The refinement lasts until the end of the scope of the test
or until \<x> may be side-effected.  For more details, see
Section~\ref{type-refinement}.


\subsection{Additional details\label{nullness-additional-details}}

The Nullness Checker does some special checks in certain circumstances, in
order to soundly reduce the number of warnings that it produces.

For example, a call to
\sunjavadoc{java/lang/System.html\#getProperty-java.lang.String-}{System.getProperty(String)}
can return null in general, but it will not return null if the argument is
one of the built-in-keys listed in the documentation of
\sunjavadoc{java/lang/System.html\#getProperties--}{System.getProperties()}.
The Nullness Checker is aware of this fact, so you do not have to suppress
a warning for a call like \<System.getProperty("line.separator")>.  The
warning is still issued for code like this:

\begin{Verbatim}
  final String s = "line.separator";
  nonNullvar = System.getProperty(s);
\end{Verbatim}

\noindent
though that case could be handled as well, if desired.
(Suppression of the warning is, strictly speaking, not sound, because a
library that your code calls, or your code itself, could perversely change
the system properties; the Nullness Checker assumes this bizarre coding
pattern does not happen.)


\subsection{Inference of \code{@NonNull} and \code{@Nullable} annotations\label{nullness-inference}}

It can be tedious to write annotations in your code.  Tools exist that
can automatically infer annotations and insert them in your source code.
(This is different than type qualifier refinement for local variables
(Section~\ref{type-refinement}), which infers a more specific type for
local variables and uses them during type-checking but does not insert them
in your source code.  Type qualifier refinement is always enabled, no
matter how annotations on signatures got inserted in your source code.)

Your choice of tool depends on what default annotation (see
Section~\ref{null-defaults}) your code uses.  You only need one of these tools.

\begin{itemize}

\item
  Inference of \refqualclass{checker/nullness/qual}{Nullable}:
  %
  If your code uses the standard CLIMB-to-top default (Section~\ref{climb-to-top}) or
  the NonNull default, then use the
  \href{http://plse.cs.washington.edu/daikon/download/doc/daikon.html#AnnotateNullable}{AnnotateNullable}
  tool of the \href{http://plse.cs.washington.edu/daikon/}{Daikon invariant
    detector}.

\item
  Inference of \refqualclass{checker/nullness/qual}{NonNull}:
  %
  If your code uses the Nullable default, use one of these tools:
\begin{itemize}
\item
  \href{https://www.juliasoft.com/eng}{Julia analyzer},
\item
  \href{http://nit.gforge.inria.fr}{Nit: Nullability Inference Tool},
\item
  \href{http://jastadd.org/jastadd-tutorial-examples/non-null-types-for-java/}{Non-null
    checker and inferencer} of the \href{http://jastadd.org/}{JastAdd
    Extensible Compiler}.
\end{itemize}

\end{itemize}



\section{Suppressing nullness warnings\label{suppressing-warnings-nullness}}

When the Nullness Checker reports a warning, it's best to change the code
or its annotations, to eliminate the warning.  Alternately, you can
suppress the warning, which does not change the code but prevents the
Nullness Checker from reporting this particular warning to you.

\begin{sloppypar}
The Checker Framework supplies several ways to suppress warnings, most
notably the \<@SuppressWarnings("nullness")> annotation (see
Chapter~\ref{suppressing-warnings}).  An example use is
\end{sloppypar}

%BEGIN LATEX
\begin{smaller}
%END LATEX
\begin{Verbatim}
    // might return null
    @Nullable Object getObject(...) { ... }

    void myMethod() {
      @SuppressWarnings("nullness") // with argument x, getObject always returns a non-null value
      @NonNull Object o2 = getObject(x);
\end{Verbatim}
%BEGIN LATEX
\end{smaller}
%END LATEX


The Nullness Checker supports an additional warning suppression key,
\<nullness:generic.argument>.
Use of \<@SuppressWarnings("nullness:generic.argument")> causes the Nullness
Checker to suppress warnings related to misuse of generic type
arguments.  One use for this key is when a class is declared to take only
\<@NonNull> type arguments, but you want to instantiate the class with a
\<@Nullable> type argument, as in \code{List<@Nullable Object>}.

The Nullness Checker also permits you to use assertions or method calls to
suppress warnings; see below.

% TODO: check whether the SuppressWarnings keys are correct.


\subsection{Suppressing warnings with assertions and method calls\label{suppressing-warnings-with-assertions}}

Occasionally, it is inconvenient or
verbose to use the \<@SuppressWarnings> annotation.  For example, Java does
not permit annotations such as \<@SuppressWarnings> to appear on statements.
In such cases, you can use the \<@AssumeAssertion> string in
an \<assert> message (see Section~\ref{assumeassertion}).

If you need to suppress a warning within an expression, then
sometimes writing an assertion is not convenient.  In such a case,
you can suppress warnings by writing a call to the
\refmethod{checker/nullness}{NullnessUtil}{castNonNull}{-T-} method.
The rest of this section discusses the \<castNonNull> method.

The Nullness Checker considers both the return value, and also the
argument, to be non-null after the \<castNonNull> method call.
The Nullness Checker issues no warnings in any of the following
code:

\begin{Verbatim}
  // One way to use castNonNull as a cast:
  @NonNull String s = castNonNull(possiblyNull1);

  // Another way to use castNonNull as a cast:
  castNonNull(possiblyNull2).toString();

  // It is possible, but not recommmended, to use castNonNull as a statement:
  // (It would be better to write an assert statement with @AssumeAssertion
  // in its message, instead.)
  castNonNull(possiblyNull3);
  possiblyNull3.toString();
\end{Verbatim}

  The \<castNonNull> method throws \<AssertionError> if Java assertions are enabled and
  the argument is \<null>.  However, it is not intended for general defensive
  programming; see Section~\ref{defensive-programming}.

  A potential disadvantage of using the \<castNonNull> method is that your
  code becomes dependent on the Checker Framework at run time as well as at
  compile time:  you need to include \<checker-qual.jar>
  or \<checker.jar> on your classpath at run time.
  If this is a problem, you can copy the implementation of
  \<castNonNull> into your own code, and possibly renaming it if you do not
  like the name.  Be sure to retain the documentation that indicates that
  your copy is intended for use only to suppress warnings and not for
  defensive programming.  See Section~\ref{defensive-programming} for an
  explanation of the distinction.

\begin{sloppypar}
The Nullness Checker introduces a new method, rather than re-using an
existing method such as \<org.junit.Assert.assertNotNull(Object)> or
\<com.google.common.base.Preconditions.checkNotNull(Object)>.  Those
methods are commonly used for defensive programming, so it is impossible to
know the programmer's intent when writing them.  Therefore, it is important to
have a method call that is used only for warning suppression.  See
Section~\ref{defensive-programming} for a discussion of
the distinction between warning suppression and defensive programming.
% Also, different checking tools issue different false warnings that
% need to be suppressed, so warning suppression needs to be customized for
% each tool.
\end{sloppypar}


\section{Examples\label{nullness-example}}

\subsection{Tiny examples\label{nullness-tiny-examples}}

To try the Nullness Checker on a source file that uses the \refqualclass{checker/nullness/qual}{NonNull} qualifier,
use the following command (where \code{javac} is the Checker Framework compiler that
is distributed with the Checker Framework):

%BEGIN LATEX
\begin{smaller}
%END LATEX
\begin{Verbatim}
  javac -processor org.checkerframework.checker.nullness.NullnessChecker docs/examples/NullnessExample.java
\end{Verbatim}
%BEGIN LATEX
\end{smaller}
%END LATEX

\noindent
Compilation will complete without warnings.

To see the checker warn about incorrect usage of annotations (and therefore the
possibility of a null pointer exception at run time), use the following command:

\begin{mysmall}
\begin{Verbatim}
  javac -processor org.checkerframework.checker.nullness.NullnessChecker docs/examples/NullnessExampleWithWarnings.java
\end{Verbatim}
\end{mysmall}


\noindent
The compiler will issue two warnings regarding violation of the semantics of
\refqualclass{checker/nullness/qual}{NonNull}.
% in the \code{NonNullExampleWithWarnings.java} file.


\subsection{Example annotated source code\label{nullness-annotated-library}}

Some libraries that are annotated with nullness qualifiers are:

\begin{itemize}
\item
The Nullness Checker itself.

\item
The
\href{https://mernst.github.io/plume-lib/}{Plume-lib library}.
Run the command \code{make check-nullness}.


\item
The
\href{http://plse.cs.washington.edu/daikon/}{Daikon invariant detector}.
Run the command \code{make check-nullness}.

% \item
% The annotation scene library.
% To run the Nullness Checker on the annotation scene library,
% % TODO: how does one do this?
% first download the scene library suite (which includes build
% dependencies for the scene library as well as its source code) and extract it
% into your Checker Framework installation. The checker can then be run on the annotation
% scene library with Apache Ant using the following commands:
%
% \begin{Verbatim}
%   cd checkers
%   ant -f scene-lib-test.xml
% \end{Verbatim}
%
% % \noindent
% % where \code{checkers} is the location of the Checker Framework installation.
%
% You can view the annotated source code, which contains \refqualclass{checker/nullness/qual}{NonNull} annotations, in
% the
% %BEGIN LATEX
% \begin{smaller}
% %END LATEX
% \code{checkers/scene-lib-test/src/annotations/}
% %BEGIN LATEX
% \end{smaller}
% %END LATEX
% directory.

\end{itemize}


\section{Tips for getting started\label{nullness-getting-started}}

Here are some tips about getting started using the Nullness Checker on a
legacy codebase.  For more generic advice (not specific to the Nullness
Checker), see Section~\ref{get-started-with-legacy-code}.

Your goal is to add \refqualclass{checker/nullness/qual}{Nullable} annotations
to the types of any variables that can be null.  (The default is to assume
that a variable is non-null unless it has a \code{@Nullable} annotation.)
Then, you will run the Nullness Checker.  Each of its errors indicates
either a possible null pointer exception, or a wrong/missing annotation.
When there are no more warnings from the checker, you are done!

We recommend that you start by searching the code for occurrences of
\code{null} in the following locations; when you find one, write the
corresponding annotation:

\begin{itemize}
\item
  in Javadoc:  add \code{@Nullable} annotations to method signatures (parameters and return types).
% Search for "\*.*\bnull\b"
\item
  \code{return null}:  add a \code{@Nullable} annotation to the return type
  of the given method.
% Search for "return null" and "return.*? null" and "return.*: null"
\item
  \code{\emph{param} == null}:  when a formal parameter is compared to
  \code{null}, then in most cases you can add a \code{@Nullable} annotation
  to the formal parameter's type
\item
  \code{\emph{TypeName} \emph{field} = null;}:  when a field is initialized to
  \code{null} in its declaration, then it needs either a
  \refqualclass{checker/nullness/qual}{Nullable} or a
  \refqualclass{checker/nullness/qual}{MonotonicNonNull} annotation.  If the field
  is always set to a non-null value in the constructor, then you can just
  change the declaration to \code{\emph{Type} \emph{field};}, without an
  initializer, and write no type annotation (because the default is
  \<@NonNull>).
\item
  declarations of \<contains>, \<containsKey>, \<containsValue>, \<equals>,
  \<get>, \<indexOf>, \<lastIndexOf>, and \<remove> (with \<Object> as the
  argument type):
  change the argument type to \<@Nullable Object>; for \<remove>, also change
  the return type to \<@Nullable Object>.
% Emacs code for the argument types:
% ;;NOT: (tags-query-replace " apply(Object " " apply(/*@Nullable*/ Object ")
% (tags-query-replace " \\(get\\|equals\\|remove\\|contains\\|containsValue\\|containsKey\\|indexOf\\|lastIndexOf\\)(Object " " \\1(/*@Nullable*/ Object ")

\end{itemize}

\noindent
You should ignore all other occurrences of \code{null} within a method
body.  In particular, you (almost) never need to annotate local variables.

Only after this step should you run \code{ant} to invoke
the Nullness Checker.  The reason is that it is quicker to search for
places to change than to repeatedly run the checker and fix the errors it
tells you about, one at a time.

Here are some other tips:
\begin{itemize}
\item
    \begin{sloppypar}
    In any file where you write an annotation such as \code{@Nullable},
    don't forget to add \code{import org.checkerframework.checker.nullness.qual.*;}.
    \end{sloppypar}
\item
    To indicate an array that can be null, write, for example: \code{int
      @Nullable []}. \\
    By contrast, \code{@Nullable Object []} means a non-null array that
    contains possibly-null objects.
\item
    If you know that a particular variable is definitely not null, but the
    Nullness Checker estimates that the variable might be null, then you can
    make the Nullness Checker trust your judgment by writing
    an assertion (see Section~\ref{assumeassertion}):
\begin{Verbatim}
assert var != null : "@AssumeAssertion(nullness)";
\end{Verbatim}
\item
    To indicate that a routine returns the same value every time it is
    called, use \refqualclass{dataflow/qual}{Pure} (see Section~\ref{type-refinement-purity}).
\item
    To indicate a method precondition (a contract stating the conditions
    under which a client is allowed to call it), you can use annotations
    such as \refqualclass{checker/nullness/qual}{RequiresNonNull} (see Section~\ref{nullness-method-annotations}).
\end{itemize}



\section{Other tools for nullness checking\label{nullness-related-work}}

\newcommand{\linktoNonNull}{\refclass{checker/nullness/qual}{NonNull}}
\newcommand{\linktoNullable}{\refclass{checker/nullness/qual}{Nullable}}

The Checker Framework's nullness annotations are similar to annotations used
in IntelliJ IDEA, FindBugs, JML, the JSR 305 proposal, NetBeans, and other
tools.  In particular, IDE tools such as Eclipse and IntelliJ should be
viewed as bug-finding tools rather than verification tools, since they
give up precision, soundness, or both in favor of being fast and
easy to use.  Also
see Section~\ref{other-tools} for a comparison to other tools.

You might prefer to use the Checker Framework because it has a more
powerful analysis that can warn you about more null pointer errors in your
code.

If your code is already annotated with a different nullness
annotation, the Checker Framework can type-check your code.
It treats annotations from other tools
exactly as if you had written the corresponding annotation from the
Nullness Checker, as described in Figure~\ref{fig-nullness-refactoring}.


% These lists should be kept in sync with NullnessAnnotatedTypeFactory.java .
\begin{figure}
\begin{center}
% The ~ around the text makes things look better in Hevea (and not terrible
% in LaTeX).
\begin{tabular}{ll}
\begin{tabular}{|l|}
\hline
 ~android.annotation.NonNull~ \\ \hline
 ~android.support.annotation.NonNull~ \\ \hline
 ~com.sun.istack.internal.NotNull~ \\ \hline
 ~edu.umd.cs.findbugs.annotations.NonNull~ \\ \hline
 ~javax.annotation.Nonnull~ \\ \hline
 ~javax.validation.constraints.NotNull~ \\ \hline
 ~lombok.NonNull~ \\ \hline
 ~org.eclipse.jdt.annotation.NonNull~ \\ \hline
 ~org.eclipse.jgit.annotations.NonNull~ \\ \hline
 ~org.jetbrains.annotations.NotNull~ \\ \hline
 ~org.jmlspecs.annotation.NonNull~ \\ \hline
 ~org.netbeans.api.annotations.common.NonNull~ \\ \hline
\end{tabular}
&
$\Rightarrow$
~org.checkerframework.checker.nullness.qual.NonNull~
\\
\
\\
\begin{tabular}{|l|l|}
\hline
 ~android.annotation.Nullable~ \\ \hline
 ~android.support.annotation.Nullable~ \\ \hline
 ~com.sun.istack.internal.Nullable~ \\ \hline
 ~edu.umd.cs.findbugs.annotations.Nullable~ \\ \hline
 ~edu.umd.cs.findbugs.annotations.CheckForNull~ \\ \hline
 ~edu.umd.cs.findbugs.annotations.UnknownNullness~ \\ \hline
 ~javax.annotation.Nullable~ \\ \hline
 ~javax.annotation.CheckForNull~ \\ \hline
 ~org.eclipse.jdt.annotation.Nullable~ \\ \hline
 ~org.eclipse.jgit.annotations.Nullable~ \\ \hline
 ~org.jetbrains.annotations.Nullable~ \\ \hline
 ~org.jmlspecs.annotation.Nullable~ \\ \hline
 ~org.netbeans.api.annotations.common.NullAllowed~ \\ \hline
 ~org.netbeans.api.annotations.common.CheckForNull~ \\ \hline
 ~org.netbeans.api.annotations.common.NullUnknown~ \\ \hline
\end{tabular}
&
$\Rightarrow$
~org.checkerframework.checker.nullness.qual.Nullable~
\end{tabular}
\end{center}
%BEGIN LATEX
\vspace{-1.5\baselineskip}
%END LATEX
\caption{Correspondence between other nullness annotations and the
  Checker Framework's annotations.}
\label{fig-nullness-refactoring}
\end{figure}

%% Removed, because it's tedious and should be obvious to a decent programmer.
% Your IDE may be able to do that for you.  Alternately, do the following:
% \begin{enumerate}
% \item
%   replace \<@Nonnull> by \<@NonNull> (note capitalization difference)
% \item
%   replace \<@CheckForNull> by \<@Nullable>
% \item
%   replace \<@UnknownNullness> by \<@Nullable>
% \item
%   convert each single-type import statement (without a ``\<*>'' character)
%    according to the table above.
% \item
%   convert each on-demand import statements, such as ``\<import
%    edu.umd.cs.findbugs.annotations.*;>''.
% \begin{itemize}
%    \item
%   One approach is to change it into a set of single-type imports,
%       then convert the relevant ones.
%    \item
%   Another approach is to change it according to the table above, then
%       try to compile and re-introduce the single-type imports as necessary.
% \end{itemize}
%    These approaches let you continue to use other annotations in the
%    \<edu.umd.cs.findbugs.annotations> package, even though you are not using
%    its nullness annotations.
% \end{enumerate}


The Checker Framework may issue more or fewer errors than another tool.
This is expected, since each tool uses a different analysis.  Remember that
the Checker Framework aims at soundness:  it aims to never miss a possible
null dereference, while at the same time limiting false reports.  Also,
note FindBugs's non-standard meaning for \<@Nullable>
(Section~\ref{findbugs-nullable}).

Java permits you to import at most one annotation of a given name.  For
example, if you use both \<android.annotation.NonNull> and
\<lombok.NonNull> in your source code, then you must write at least one of
them in fully-qualified form, as \<@android.annotation.NonNull> rather than
as \<@NonNull>.

Note that some older tools interpret array and varargs declarations
inconsistently with the Java specification.  For example, they might
interpret \<@NonNull Object []> as ``non-null array of objects'', rather
than as ``array of non-null objects'' which is the correct Java
interpretation.  Such an interpretation is unfortunate and confusing.  See
Section~\ref{faq-array-syntax-meaning} for some more details about this
issue.


\subsection{Which tool is right for you?\label{choosing-nullness-tool}}

Different tools are appropriate in different circumstances.  Here is a
brief comparison with FindBugs, but similar points apply to other tools.

The Checker Framework has a more powerful nullness analysis; FindBugs misses
some real
errors.  FindBugs requires you to annotate your code, but usually not as
thoroughly as the Checker Framework does.  Depending on the importance of
your code, you may desire:  no nullness checking, the cursory checking of
FindBugs, or the thorough checking of the Checker Framework.  You might
even want to ensure that both tools run, for example if your coworkers or
some other organization are still using FindBugs.  If you know that you
will eventually want to use the Checker Framework, there is no point using
FindBugs first; it is easier to go straight to using the Checker Framework.

FindBugs can find other errors in addition to nullness errors; here
we focus on its nullness checks.  Even if you use FindBugs for its other
features, you may want to use the Checker Framework for analyses that can
be expressed as pluggable type-checking, such as detecting nullness errors.

Regardless of whether you wish to use the FindBugs nullness analysis, you
may continue running all of the other FindBugs analyses at the same time as
the Checker Framework; there are no interactions among them.

If FindBugs (or any other tool) discovers a nullness error that the Checker
Framework does not, please report it to us (see
Section~\ref{reporting-bugs}) so that we can enhance the Checker Framework.



\subsection{Incompatibility note about FindBugs \tt{@Nullable}\label{findbugs-nullable}}

FindBugs has a non-standard definition of \<@Nullable>.  FindBugs's treatment is not
documented in its own
\href{http://findbugs.sourceforge.net/api/edu/umd/cs/findbugs/annotations/Nullable.html}{Javadoc};
it is different from the definition of \<@Nullable> in every other tool for
nullness analysis; it means the same thing as \<@NonNull> when applied to a
formal parameter; and it invariably surprises programmers.  Thus, FindBugs's
\<@Nullable> is detrimental rather than useful as documentation.
In practice, your best bet is to not rely on FindBugs for nullness analysis,
even if you find FindBugs useful for other purposes.

You can skip the rest of this section unless you wish to learn more details.

FindBugs suppresses all warnings at uses of a \<@Nullable> variable.
(You have to use \<@CheckForNull> to
indicate a nullable variable that FindBugs should check.)  For example:

\begin{Verbatim}
     // declare getObject() to possibly return null
     @Nullable Object getObject() { ... }

     void myMethod() {
       @Nullable Object o = getObject();
       // FindBugs issues no warning about calling toString on a possibly-null reference!
       o.toString();
     }
\end{Verbatim}

\noindent
The Checker Framework does not emulate this non-standard behavior of
FindBugs, even if the code uses FindBugs annotations.

With FindBugs, you annotate a declaration, which suppresses checking at
\emph{all} client uses, even the places that you want to check.
It is better to suppress warnings at only the specific client uses
where the value is known to be non-null; the Checker Framework supports
this, if you write \<@SuppressWarnings> at the client uses.
The Checker Framework also supports suppressing checking at all client uses,
by writing a \<@SuppressWarnings> annotation at the declaration site.
Thus, the Checker Framework supports both use cases, whereas FindBugs
supports only one and gives the programmer less flexibility.

In general, the Checker Framework will issue more warnings than FindBugs,
and some of them may be about real bugs in your program.
See Section~\ref{suppressing-warnings-nullness} for information about
suppressing nullness warnings.

(FindBugs made a poor choice of names.  The choice of names should make a
clear distinction between annotations that specify whether a reference is
null, and annotations that suppress false warnings.  The choice of names
should also have been consistent for other tools, and intuitively clear to
programmers.  The FindBugs choices make the FindBugs annotations less
helpful to people, and much less useful for other tools.  As a separate
issue, the FindBugs
analysis is also very imprecise.  For type-related analyses, it is best to
stay away from the FindBugs nullness annotations, and use a more capable
tool like the Checker Framework.)



% As background, here is an explanation of the (sometimes surprising)
% semantics of the FindBugs nullness annotations.
%
%  * edu.umd.cs.findbugs.annotations.NonNull     javax.annotation.Nonnull
%    These mean the obvious thing:   the reference is never null.
%
%  * edu.umd.cs.findbugs.annotations.Nullable     javax.annotation.Nullable
%    This means that the value may be null, but that *all warnings should be
%    suppressed* regarding its use.  In other words, the value is really
%    nullable, but clients should treat it as non-null.  For example:
%
%      // declare getObject() to possibly return null
%      @Nullable Object getObject() { ... }
%
%      // FindBugs issues no warning about calling toString on a possibly-null reference
%      getObject().toString();
%
%    In the Checker Framework, this corresponds to declaring the method
%    return value as @Nullable, then suppressing warnings at client uses
%    that are known to be non-null.  An easy way to suppress the warning
%    is by adding an assert statement about the return value.
%
%    (Alternately, you could declare the method return value as @NonNull, and
%    suppress warnings within the method definition where it returns null,
%    but this approach is not recommended because the @NonNull annotation on
%    the return value would be misleading, and warnings should be suppressed
%    at particular sites where they are known to be unnecessary, not at all
%    call sites whatsoever.)
%
%  * edu.umd.cs.findbugs.annotations.CheckForNull      javax.annotation.CheckForNull
%    This means that the value may be null.  To avoid a NullPointerException,
%    every client should check nullness before dereferencing the value.
%    In the Checker Framework, this corresponds to @Nullable.


\subsection{Relationship to \tt{Optional<T>}\label{nullness-vs-optional}}

Many null pointer exceptions occur because the programmer forgets to check
whether a reference is null before dereferencing it.  Java 8's
\sunjavadoc{java/util/Optional.html}{\code{Optional<T>}}
class provides a partial solution:  you cannot
dereference the contained value without calling the \<get> method.

However, the use of \code{Optional} for this purpose is unsatisfactory.
First, it adds syntactic complexity, making your code longer and harder to
read.  (The \code{Optional} class provides some operations, such as \<map>
and \<orElse>, that you would otherwise have to write; without these its
code bloat would be even worse.)  Second, there is no guarantee that the
programmer remembers to call \<isPresent> before calling \<get>.  Thus, use
of \code{Optional} doesn't solve the underlying problem --- it merely
converts a \code{NullPointerException} into a
\code{NoSuchElementException} exception, and in either case your code
crashes.

The Nullness Checker does not suffer these limitations.  It works with
existing code and types, it ensures that you check for null wherever
necessary, and it infers when the check for null is not necessary based on
previous statements in the method.

See the article
\href{http://homes.cs.washington.edu/~mernst/advice/nothing-is-better-than-optional.html}{``Nothing
  is better than Java's \<Optional> class''} for more details and
explanation of the
\href{http://homes.cs.washington.edu/~mernst/advice/nothing-is-better-than-optional.html#benefits-of-nullable}{benefits of \<@Nullable> over \<Optional>}.

Java's \<Optional> class provides utility routines to reduce clutter when
using \<Optional>.  The Nullness Checker provides an
\refclass{checker/nullness}{Opt} class that provides all the same methods,
but written for regular possibly-null Java references.


\section{Initialization Checker\label{initialization-checker}}

The Initialization Checker determines whether an object is initialized or
not.  For any object that is not fully initialized, the Nullness Checker
treats its fields as possibly-null --- even fields annotated as
\<@NonNull>.

Every object's fields start out as null.  By the time the constructor
finishes executing, the \<@NonNull> fields have been set to a different
value.  Your code can suffer a NullPointerException when using a
\<@NonNull> field, if your code uses the field during initialization.
The Nullness Checker prevents this problem by warning you anytime that you
may be accessing an uninitialized field.  This check is useful because it
prevents errors in your code.  However, the analysis can be confusing to
understand.  If you wish to disable the initialization checks, pass the
command-line argument \<-AsuppressWarnings=uninitialized> when running the
Nullness Checker.  You will no longer get a guarantee of no null pointer
exceptions, but you can still use the Nullness Checker to find most of the
null pointer problems in your code.


An object is partially initialized from the time that its constructor starts until its constructor
finishes.  This is relevant to the Nullness Checker because while the
constructor is executing --- that is, before initialization completes ---
a \<@NonNull>
field may be observed to be null, until that field is set.  In
particular, the Nullness Checker issues a warning for code like this:

\begin{Verbatim}
  public class MyClass {
    private @NonNull Object f;
    public MyClass(int x, int y) {
      // Error because constructor contains no assignment to this.f.
      // By the time the constructor exits, f must be initialized to a non-null value.
    }
    public MyClass(int x) {
      // Error because this.f is accessed before f is initialized.
      // At the beginning of the constructor's execution, accessing this.f
      // yields null, even though field f has a non-null type.
      this.f.toString();
    }
    public MyClass(int x, int y, int z) {
      m();
    }
    public void m() {
      // Error because this.f is accessed before f is initialized,
      // even though the access is not in a constructor.
      // When m is called from the constructor, accessing f yields null,
      // even though field f has a non-null type.
      this.f.toString();
    }
\end{Verbatim}

\noindent
When a field \<f> is declared with a \refqualclass{checker/nullness/qual}{NonNull}
type, then code can depend on the fact that the field is not \<null>.
However, this guarantee does not hold for a partially-initialized object.

The Nullness Checker uses three annotations to indicate whether an object
is initialized (all its \<@NonNull> fields have been assigned), under
initialization (its constructor is currently executing), or its
initialization state is unknown.

These distinctions are mostly relevant within the constructor, or for
references to \code{this} that escape the constructor (say, by being stored
in a field or passed to a method before initialization is complete).
Use of initialization annotations is rare in most code.

\label{underinitialization-examples}

The most common use for the \<@UnderInitialization> annotation is for a
helper routine that is called by constructor.  For example:

\begin{Verbatim}
  class MyClass {
    Object field1;
    Object field2;
    Object field3;

    public MyClass(String arg1) {
      this.field1 = arg1;
      init_other_fields();
    }

    // A helper routine that initializes all the fields other than field1.
    @EnsuresNonNull({"field2", "field3"})
    private void init_other_fields(@UnderInitialization(Object.class) MyClass this) {
      field2 = new Object();
      field3 = new Object();
    }

    public MyClass(String arg1, String arg2, String arg3) {
      this.field1 = arg1;
      this.field2 = arg2;
      this.field3 = arg3;
      checkRep();
    }

    // Verify that the representation invariants are satisfied.
    // Works as long as the MyClass fields are initialized, even if the reciever's
    // class is a subclass of MyClass and not all of the subclass fields are initialized.
    private void checkRep(@UnderInitialization(MyClass.class) MyClass this) {
      ...
    }


  }
\end{Verbatim}

For compatibility with Java 5/6/7, you can write the receiver
parameter in comments (see Section~\ref{receivers-and-imports-in-comments}):
\begin{Verbatim}
    private void init_other_fields(/*>>>@UnderInitialization(Object.class) MyClass this*/) {
\end{Verbatim}

% Most readers can
% skip this section on first reading; you can return to it once you have
% mastered the rest of the Nullness Checker.

Note that it would not be sound to type {@code this} as fully {@link Initialized} anywhere in
a constructor (with the exception of final classes, for which the set
of all fields is known),
because there might be subclasses with uninitialized fields. The following example shows why:

\begin{Verbatim}
class A {
   @NonNull String a;
   public A() {
       a = "";
       // Now, all fields of A are initialized.
       // However, if this constructor is invoked as part of 'new B()', then
       // the fields of B are not yet initialized.
       // If we would type 'this' as @Initialized, then the following call is valid:
       foo();
   }
   void foo() {}
}

class B extends A {
   @NonNull String b;
   @Override
   void foo() {
       // Dereferencing 'b' is ok, since 'this' is @Initialized and 'b' @NonNull.
       // However, when executing 'new B()', this line throws a null-pointer exception.
       b.toString();
   }
}
\end{Verbatim}


\subsection{Initialization qualifiers\label{initialization-qualifiers}}

\begin{figure}
\includeimage{initialization}{4cm}
\caption{Partial type hierarchy for the Initialization type system.
  \<@UnknownInitialization> and \<@UnderInitialization> each take an
  optional parameter indicating how far initialization has proceeded, and
  the right side of the figure illustrates its type hierarchy in more detail.}
\label{fig-initialization-hierarchy}
\end{figure}

The initialization hierarchy is shown in Figure~\ref{fig-initialization-hierarchy}.
The initialization hierarchy contains these qualifiers:

\begin{description}

\item[\refqualclass{checker/initialization/qual}{Initialized}]
  indicates a type that contains a fully-initialized object.  \code{Initialized}
  is the default, so there is little need for a programmer to write this
  explicitly.

\item[\refqualclass{checker/initialization/qual}{UnknownInitialization}]
  indicates a type that may contain a partially-initialized object.  In a
  partially-initialized object, fields that are annotated as
  \refqualclass{checker/nullness/qual}{NonNull} may be null because the field
  has not yet been assigned.

  \<@UnknownInitialization> takes a parameter that is the class the object
  is definitely initialized up to.  For instance, the type
  \<@UnknownInitialization(Foo.class)> denotes an object in which every
  fields declared in \<Foo> or its superclasses is initialized, but other
  fields might not be.
  Just \<@UnknownInitialization> is equivalent to
  \<@UnknownInitialization(Object.class)>.

\item[\refqualclass{checker/initialization/qual}{UnderInitialization}]
  indicates a type that contains a partially-initialized object that is
  under initialization --- that is, its constructor is currently executing.
  It is otherwise the same as \<@UnknownInitialization>.  Within the
  constructor, \code{this} has
  \refqualclass{checker/initialization/qual}{UnderInitialization} type until
  all the \code{@NonNull} fields have been assigned.

\end{description}

  A partially-initialized object (\code{this} in a constructor) may be
  passed to a helper method or stored in a variable; if so, the method
  receiver, or the field, would have to be annotated as
  \<@UnknownInitialization> or as \<@UnderInitialization>.

% However, if the constructor makes
% a method call (passing \code{this} as a parameter or the receiver), then
% the called method could observe the object in an illegal state.

If a reference has
\code{@UnknownInitialization} or \code{@UnderInitialization} type, then all of its \code{@NonNull} fields are treated as
\refqualclass{checker/nullness/qual}{MonotonicNonNull}:  when read, they are
treated as being \refqualclass{checker/nullness/qual}{Nullable}, but when
written, they are treated as being
\refqualclass{checker/nullness/qual}{NonNull}.

\begin{sloppypar}
The initialization hierarchy is orthogonal to the nullness hierarchy.  It
is legal for a reference to be \<@NonNull @UnderInitialization>, \<@Nullable @UnderInitialization>,
\<@NonNull @Initialized>, or \<@Nullable @Initialized>.  The nullness hierarchy tells
you about the reference itself:  might the reference be null?  The initialization
hierarchy tells you about the \<@NonNull> fields in the referred-to object:
might those fields be temporarily null in contravention of their
type annotation?
% It's a figure rather than appearing inline so as not to span page breaks
% in the printed manual.
Figure~\ref{fig-initialization-examples} contains some examples.
\end{sloppypar}

\begin{figure}
\begin{tabular}{l|l|l}
Declarations & Expression & Expression's nullness type, or checker error \\ \hline
\begin{minipage}{1.5in}
\begin{Verbatim}
class C {
  @NonNull Object f;
  @Nullable Object g;
  ...
}
\end{Verbatim}
\end{minipage} & & \\ \cline{2-3}
\<@NonNull @Initialized C a;>
& \<a> & \<@NonNull> \\
& \<a.f> & \<@NonNull> \\
& \<a.g> & \<@Nullable> \\ \cline{2-3}
\<@NonNull @UnderInitialization C b;>
& \<b> & \<@NonNull> \\
& \<b.f> & \<@MonotonicNonNull> \\
& \<b.g> & \<@Nullable> \\ \cline{2-3}
\<@Nullable @Initialized C c;>
& \<c> & \<@Nullable> \\
& \<c.f> & error: deref of nullable \\
& \<c.g> & error: deref of nullable \\ \cline{2-3}
\<@Nullable @UnderInitialization C d;>
& \<d> & \<@Nullable> \\
& \<d.f> & error: deref of nullable \\
& \<d.g> & error: deref of nullable \\
\end{tabular}
\caption{Examples of the interaction between nullness and initialization.
  Declarations are shown at the left for reference, but the focus of the
  table is the expressions and their nullness type or error.}
\label{fig-initialization-examples}
\end{figure}


% Does our implementation handle static fields soundly?  NO!  See issue
% #105.  Maybe document this?


\subsection{How an object becomes initialized\label{becoming-initialized}}

Within the constructor,
\code{this} starts out with \refqualclass{checker/initialization/qual}{UnderInitialization} type.
As soon as all of the \refqualclass{checker/nullness/qual}{NonNull} fields
in class $C$ have been initialized, then \code{this} is treated as
\refqualclass{checker/initialization/qual}{UnderInitialization}\code{(\emph{C})}.
This means that \code{this} is still being initialized, but all
initialization of $C$'s fields is complete, including all fields of supertypes.
Eventually, when all constructors complete, the type is
\refqualclass{checker/initialization/qual}{Initialized}.

The Initialization Checker issues an error if the constructor fails to initialize
any \code{@NonNull} field.  This ensures that the object is in a legal (initialized)
state by the time that the constructor exits.
This is different than Java's test for definite assignment (see
\href{https://docs.oracle.com/javase/specs/jls/se8/html/jls-16.html}{JLS ch.16}),
% , which requires that local
% variables (and blank \code{final} fields) must be assigned.  Java does not
% require that non-\code{final} fields be assigned, since
which does not apply to fields (except blank final ones, defined in
\href{https://docs.oracle.com/javase/specs/jls/se8/html/jls-4.html#jls-4.12.4}{JLS \S 4.12.4}) because fields
have a default value of null.


All \code{@NonNull} fields must either have a
default in the field declaration, or be assigned in the constructor or in a
helper method that the constructor calls.  If
your code initializes (some) fields in a helper method, you will need to
annotate the helper method with an annotation such as
\refqualclass{checker/nullness/qual}{EnsuresNonNull}\code{(\{"field1", "field2"\})}
for all the fields that the helper method assigns.
It's a bit odd, but you use that same annotation, \code{@EnsuresNonNull},
to indicate that a primitive field has its value set in a helper method,
which is relevant when you supply the \code{-Alint=uninitialized}
command-line option (see Section~\ref{lint-uninitialized-section}).

% TODO:
% We need an
%   @EnsuresInitialized("b")
% that is analogous to
%   @EnsuresNonNull("b")
% when we are dealing with a field of primitive type.
% But, for now just use the same annotation, @EnsuresNonNull, for both purposes.


\subsection{Partial initialization\label{partial-initialization}}

So far, we have discussed initialization as if it is an all-or-nothing property:
an object is non-initialized until initialization completes, and then it is initialized.  The full truth is a bit more complex:  during the
initialization process an object can be partially initialized, and as the
object's superclass constructors complete, its initialization status is updated.  The
Initialization Checker lets you express such properties when necessary.

Consider a simple example:

\begin{Verbatim}
class A {
  Object aField;
  A() {
    aField = new Object();
  }
}
class B extends A {
  Object bField;
  B() {
    super();
    bField = new Object();
  }
}
\end{Verbatim}

Consider what happens during execution of \<new B()>.

\begin{enumerate}
\item \<B>'s constructor begins to execute.  At this point, neither the
  fields of \<A> nor those of \<B> have been initialized yet.
\item \<B>'s constructor calls \<A>'s constructor, which begins to execute.
  No fields of \<A> nor of \<B> have been initialized yet.
\item \<A>'s constructor completes.  Now, all the fields of \<A> have been
  initialized, and their invariants (such as that field \<a> is non-null) can be
  depended on.  However, because \<B>'s constructor has not yet completed
  executing, the object being constructed is not yet fully initialized.
  When treated as an \<A> (e.g., if only the \<A> fields are accessed), the
  object is initialized, but when treated as a \<B>, the object
  is still non-initialized.
\item \<B>'s constructor completes.  The object is initialized when treated
  as an \<A> or a \<B>.  (And, the object is fully initialized
   if \<B>'s constructor was invoked via a \<new B()>.  But the type system
   cannot assume that -- there might be a \<class C extends B \{
  ... \}>, and \<B>'s constructor might have been invoked from that.)
\end{enumerate}

At any moment during initialization, the superclasses of a given class
can be divided into those that have completed initialization and those that
have not yet completed initialization.  More precisely, at any moment there
is a point in the class hierarchy such that all the classes above that
point are fully initialized, and all those below it are not yet
initialized.  As initialization proceeds, this dividing line between the
initialized and uninitialized classes moves down the type hierarchy.

The Nullness Checker lets you indicate where the dividing line is between
the initialized and non-initialized classes.
The \<@UnderInitialization(\emph{classliteral})>
indicates the first class that is known to be fully initialized.
When you write \refqualclass{checker/initialization/qual}{UnderInitialization}\code{(OtherClass.class) MyClass x;}, that
means that variable \<x> is initialized for \<OtherClass> and its
superclasses, and \<x> is (possibly) uninitialized for \<MyClass> and all subclasses.

The example above lists 4 moments during construction.  At those moments,
the type of the object being constructed is:

\begin{enumerate}
\item
  \<@UnderInitialization B>
\item
  \<@UnderInitialization A>
\item
  \<@UnderInitialization(A.class) A>
  %% Not quite equivalent because the Java (non-qualified) type differs
  % ; equivalently, \<@UnderInitialization B>
\item
  \<@UnderInitialization(B.class) B>
\end{enumerate}


\subsection{Method calls from the constructor\label{initialization-constructor}}

Consider the following incorrect program.

\begin{Verbatim}
class A {
  Object aField;
  A() {
    aField = new Object();
    process(5);  // illegal call
  }
  public void process(int arg) { ... }
}
\end{Verbatim}

The call to \<process()> is not legal.
\<process()> is declared to be called on a fully-initialized receiver, which is
the default if you do not write a different initialization annotation.
At the call to \<process()>, all the fields of \<A> have been set,
but \<this> is not fully initialized because fields in subclasses of \<A> have
not yet been set.  The type of \<this> is \<@UnderInitialization(A.class)>,
meaning that \<this> is partially-initialized, with the \<A> part of
initialization done but the initialization of subclasses not yet complete.

The Initialization Checker output indicates this problem:

\begin{Verbatim}
Client.java:7: error: [method.invocation.invalid] call to process(int) not allowed on the given receiver.
    process(5);  // illegal call
           ^
  found   : @UnderInitialization(A.class) A
  required: @Initialized A
\end{Verbatim}

Here is a subclass and client code that crashes with a \<NullPointerException>.

\begin{Verbatim}
class B extends A {
  List<Integer> processed;
  B() {
    super();
    processed = new ArrayList<Integer>();
  }
  @Override
  public void process(int arg) {
    super();
    processed.add(arg);
  }
}
class Client {
  public static void main(String[] args) {
    new B();
  }
}
\end{Verbatim}

You can correct the problem in multiple ways.

One solution is to not call methods that can be overridden from the
constructor:  move the call to \<process()> to after the constructor has
completed.

Another solution is to change the declaration of \<process()>:

\begin{Verbatim}
  public void process(@UnderInitialization(A.class) A this, int arg) { ... }
\end{Verbatim}

If you choose this solution, you will need to rewrite the definition of
\<B.process()> so that it is consistent with the declared receiver type.

A non-solution is to prevent subclasses from overriding \<process()> by
using \<final> on the method.  This doesn't work because even if
\<process()> is not overridden, it might call other methods that are
overridden.

As final classes cannot have subclasses, they can be handled more
flexibly: once all fields of the final class have been
initialized, \<this> is fully initialized.

% TODO: One could imagine having an annotation indicating that the routine
% is "truly final":  it is final and it never calls, directly, or
% indirectly, any routine that might be overridden.  I'm not sure how you
% would confirm that, given the existence of callbacks from other code.



\subsection{Initialization of circular data structures\label{circular-initialization}}

There is one final aspect of the initialization type system to be
considered:  the rules governing reading and writing to objects that are
currently under initialization (both reading from fields of objects under
initialization, as well as writing objects under initialization to fields).
By default, only fully-initialized objects can be stored in a field of
another object.  If this was the only option, then it would not be possible
to create circular data structures (such as a doubly-linked list) where
fields have a \refqualclass{checker/nullness/qual}{NonNull} type.  However,
the annotation
\refqualclass{checker/initialization/qual}{NotOnlyInitialized} can be used to
indicate that a field can store objects that are currently under initialization.
In this case, the rules for reading and writing to that field become a little
bit more interesting, to soundly support circular structures.

The rules for reading from a
\refqualclass{checker/initialization/qual}{NotOnlyInitialized} field
are summarized in Figure~\ref{fig-init-read-field}.  Essentially, nothing is
known about the initialization status of the value returned unless the receiver
was \refqualclass{checker/initialization/qual}{Initialized}.

\begin{figure}
\centering
\begin{tabular}{l|l|l}
  \<x.f>&\<f> is \<@NonNull>& \<f> is \<@Nullable>\\ \hline
  \<x> is \<@Initialized> & \<@Initialized @NonNull> & \<@Initialized @Nullable> \\
  \<x> is \<@UnderInitialization> & \<@UnknownInitialization @Nullable> & \<@UnknownInitialization @Nullable> \\
  \<x> is \<@UnknownInitialization> & \<@UnknownInitialization @Nullable> & \<@UnknownInitialization @Nullable> \\
\end{tabular}
\caption{Initialization rules for reading a \refqualclass{checker/initialization/qual}{NotOnlyInitialized} field \<f>.}
\label{fig-init-read-field}
\end{figure}

Similarly, Figure~\ref{fig-init-write-field} shows under which conditions
an assignment \<x.f = y> is allowed for a
\refqualclass{checker/initialization/qual}{NotOnlyInitialized} field \<f>.
If the receiver \<x> is
\refqualclass{checker/initialization/qual}{UnderInitialization}, then any
\<y> can be of any initialization state.  If \<y> is known to be
fully initialized, then any receiver is allowed.  All other assignments
are disallowed.

\begin{figure}
\centering
\begin{tabular}{l|ccc}
  \<x.f = y>&\<y> is \<@Initialized>& \<y> is \<@UnderInitialization>& \<y> is \<@UnknownInitialization>\\ \hline
  \<x> is \<@Initialized> & yes & no & no \\
  \<x> is \<@UnderInitialization> & yes & yes & yes \\
  \<x> is \<@UnknownInitialization> & yes & no & no \\
\end{tabular}
\caption{Rules for deciding when an assignment \<x.f = y> is allowed for a
\refqualclass{checker/initialization/qual}{NotOnlyInitialized} field \<f>.}
\label{fig-init-write-field}
\end{figure}

These rules allow for the safe initialization of circular structures.  For instance,
consider a doubly linked list:

\begin{Verbatim}
  class List<T> {
    @NotOnlyInitialized
    Node<T> sentinel;

    public List() {
      this.sentinel = new Node<T>(this);
    }

    void insert(@Nullable T data) {
      this.sentinel.insertAfter(data);
    }

    public static void main() {
      List<Integer> l = new List<Integer>();
      l.insert(1);
      l.insert(2);
    }
  }

  class Node<T> {
    @NotOnlyInitialized
    Node<T> prev;

    @NotOnlyInitialized
    Node<T> next;

    @NotOnlyInitialized
    List parent;

    @Nullable
    T data;

    // for sentinel construction
    Node(@UnderInitialization List parent) {
      this.parent = parent;
      this.prev = this;
      this.next = this;
    }

    // for data node construction
    Node(Node<T> prev, Node<T> next, @Nullable T data) {
      this.parent = prev.parent;
      this.prev = prev;
      this.next = next;
      this.data = data;
    }

    void insertAfter(@Nullable T data) {
      Node<T> n = new Node<T>(this, this.next, data);
      this.next.prev = n;
      this.next = n;
    }
  }
\end{Verbatim}

% \paragraph{Example\label{initialization-example}}
%
% As another example, consider the following 12 declarations, where class A
% extends Object and class B extends A:
%
% \begin{Verbatim}
%     @UnderInitialization(Object.class) Object uOo;
%     @Initialized Object o;
%
%     @UnderInitialization(Object.class) A uOa;
%     @UnderInitialization(A.class) A uAa;
%     @Initialized A nraA;
%
%     @UnderInitialization(Object.class) B uOb;
%     @UnderInitialization(A.class) B uAb;
%     @UnderInitialization(B.class) B uBb;
%     @Initialized B b;
% \end{Verbatim}
%
% In the following table, the type in cell C1 is a supertype of the type in
% cell C2 if:  C1 is at least as high and at least as far left in the table
% as C2 is.  For example, \<nraA>'s type is a supertype of those of \<rB>,
% \<nraB>, \<nrbB>, \<a>, and \<b>.  (The empty cells on the top row are real
% types, but are not expressible.  The other empty cells are not interesting
% types.)
%
% \noindent
% \begin{tabular}{|c|c|c|}
%
% \hline
%     \<@UnderInitialization Object rO;>
% &
% &
% \\
% \hline
%
%     \<@Initialized(Object.class) Object nroO;>
% &
% \begin{minipage}{2in}
% \begin{Verbatim}
% @UnderInitialization A rA;
% @Initialized(Object.class) A nroA;
% \end{Verbatim}
% \end{minipage}
% &
%     \<@Initialized(Object.class) B nroB;>
% \\
% \hline
%
% &
%     \<@Initialized(A.class) A nraA;>
% &
% \begin{minipage}{1.75in}
% \begin{Verbatim}
% @UnderInitialization B rB;
% @Initialized(A.class) B nraB;
% \end{Verbatim}
% \end{minipage}
% \\
% \hline
%
% &
% &
%     \<@Initialized(B.class) B nrbB;>
% \\
% \hline
%
%     \<Object o;>
% &
%     \<A a;>
% &
%     \<B b;>
% \\
% \hline
% \end{tabular}



% \urldef{\jlsconstructorbodyurl}{\url}{https://docs.oracle.com/javase/specs/jls/se8/html/jls-8.html#jls-8.8.7}
% (Recall that the superclass constructor is called on the first line, or is
% inserted automatically by the compiler before the first line, see
% \href{\jlsconstructorbodyurl}{JLS \S8.8.7}.)



\subsection{How to handle warnings\label{initialization-warnings}}

There are several ways to address a warning ``error:  the constructor does
not initialize fields: \ldots''.
\begin{itemize}
\item
  Declare the field as \refqualclass{checker/nullness/qual}{Nullable}.  Recall
  that if you did not write an annotation, the field defaults to
  \refqualclass{checker/nullness/qual}{NonNull}.
\item
  Declare the field as \refqualclass{checker/nullness/qual}{MonotonicNonNull}.
  This is appropriate if the field starts out as \<null> but is later set
  to a non-null value.  You may then wish to use the
  \refqualclass{checker/nullness/qual}{EnsuresNonNull} annotation to indicate
  which methods set the field, and the
  \refqualclass{checker/nullness/qual}{RequiresNonNull} annotation to indicate
  which methods require the field to be non-null.
\item
  Initialize the field in the constructor or in the field's initializer, if
  the field should be initialized.  (In this case, the Initialization
  Checker has found a bug!)

  Do \emph{not} initialize the field to an arbitrary non-null value just to
  eliminate the warning.  Doing so degrades your code:  it introduces a
  value that will confuse other programmers, and it converts a clear
  NullPointerException into a more obscure error.
\end{itemize}

If your code calls an instance method from a constructor, you may see a
message such as the following:

\begin{Verbatim}
Foo.java:123: error: call to initHelper() not allowed on the given receiver.
    initHelper();
              ^
  found   : @UnderInitialization(com.google.Bar.class) @NonNull MyClass
  required: @Initialized @NonNull MyClass
\end{Verbatim}

The problem is that the current object (\<this>) is under initialization,
but the receiver formal parameter (Section~\ref{faq-receiver}) of method
\<initHelper()> is implicitly annotated as
\refqualclass{checker/initialization/qual}{Initialized}.  If
\<initHelper()> doesn't depend on its receiver being initialized --- that
is, it's OK to call \<x.initHelper> even if \<x> is not initialized ---
then you can indicate that:

\begin{Verbatim}
class MyClass {
  void initHelper(@UnknownInitialization MyClass this, String param1) { ... }
}
\end{Verbatim}

\noindent
If you are using annotations in comments, you would write:

\begin{Verbatim}
class MyClass {
  void initHelper(/*>>>@UnknownInitialization MyClass this,*/ String param1) { ... }
}
\end{Verbatim}

\noindent
You are likely to want to annotate \<initHelper()> with
\refqualclass{checker/nullness/qual}{EnsuresNonNull} as well; see
Section~\ref{nullness-method-annotations}.


You may get the ``call to \ldots\ is not allowed on the given receiver''
error even if your constructor has already initialized all the fields.
For this code:

\begin{Verbatim}
public class MyClass {
  @NonNull Object field;
  public MyClass() {
    field = new Object();
    helperMethod();
  }
  private void helperMethod() {
  }
}
\end{Verbatim}

\noindent
the Nullness Checker issues the following warning:

\begin{Verbatim}
MyClass.java:7: error: call to helperMethod() not allowed on the given receiver.
    helperMethod();
                ^
  found   : @UnderInitialization(MyClass.class) @NonNull MyClass
  required: @Initialized @NonNull MyClass
1 error
\end{Verbatim}

\begin{sloppypar}
The reason is that even though the object under construction has had all
the fields declared in \<MyClass> initialized, there might be a subclass of
\<MyClass>.  Thus, the receiver of \<helperMethod> should be declared as
\<@UnderInitialization(MyClass.class)>, which says that initialization has
completed for all the \<MyClass> fields but may not have been completed
overall.  If \<helperMethod> had been a public method that could also be called after
initialization was actually complete, then the receiver should have type
\<@UnknownInitialization>, which is the supertype of
\<@UnknownInitialization> and \<@UnderInitialization>.
\end{sloppypar}


\subsection{More details about initialization checking\label{initialization-checking}}


\paragraph{Suppressing warnings\label{initialization-checking-suppressing-warnings}}

\begin{sloppypar}
You can suppress warnings related to partially-initialized objects with
\<@SuppressWarnings("initialization")>.
This can be placed on a single field; on a constructor; or on a class to
suppress all initialization warnings for all constructors.
\end{sloppypar}

To disable initialization checking, supply the command-line argument
\<-AsuppressWarnings=uninitialized>.  Do \emph{not} use
\<-AsuppressWarnings=initialization>, because doing so will disable all
nullness checking as well as all initialization checking.  (That is because
of an implementation detail of the Nullness and Initialization Checkers:
they are actually the same checker, rather than being two separate checkers
that are aggregated together.)


\paragraph{Checking initialization of all fields, not just \code{@NonNull} ones\label{initialization-checking-all-fields}}

When the \code{-Alint=uninitialized} command-line option is provided, then
an object is considered uninitialized until \emph{all} its fields are assigned, not
just the \code{@NonNull} ones.  See Section~\ref{lint-uninitialized-section}.


\paragraph{Use of method annotations\label{initialization-checking-method-annotations}}

A method with a non-initialized receiver may assume that a few fields (but not all
of them) are non-null, and it sometimes sets some more fields to non-null
values.  To express these concepts, use the
\refqualclass{checker/nullness/qual}{RequiresNonNull},
\refqualclass{checker/nullness/qual}{EnsuresNonNull}, and
\refqualclass{checker/nullness/qual}{EnsuresNonNullIf} method annotations;
see Section~\ref{nullness-method-annotations}.


\paragraph{Source of the type system\label{initialization-checking-type-system}}

The type system enforced by the Initialization Checker is known as
``Freedom Before Commitment''~\cite{SummersM2011}.  Our implementation
changes its initialization modifiers (``committed'', ``free'', and
``unclassified'') to ``initialized'', ``unknown initialization'', and
``under initialization''.  Our implementation also has several
enhancements.  For example, it supports partial initialization (the
argument to the \<@UnknownInitialization> and \<@UnderInitialization>
annotations.



\subsection{Rawness Initialization Checker\label{initialization-rawness-checker}}

The Checker Framework supports two different initialization checkers that
are integrated with the Nullness Checker.
You can use whichever one you prefer.

One (described in most of Section~\ref{initialization-checker}) uses the three annotations
\refqualclass{checker/initialization/qual}{Initialized},
\refqualclass{checker/initialization/qual}{UnknownInitialization}, and
\refqualclass{checker/initialization/qual}{UnderInitialization}.
We recommend that you use it.

The other (described here in Section~\ref{initialization-rawness-checker})
uses the two annotations
\refqualclass{checker/nullness/qual}{Raw} and
\refqualclass{checker/nullness/qual}{NonRaw}.
The rawness type system is slightly easier to use but slightly less
expressive.

\begin{sloppypar}
To run the Nullness Checker with the rawness variant of the Initialization
Checker, invoke the NullnessRawnessChecker rather than the NullnessChecker;
that is, supply the \code{-processor org.checkerframework.checker.nullness.NullnessRawnessChecker}
command-line option to javac.
Although \code{@Raw} roughly corresponds to \code{@UnknownInitialization}
and \code{@NonRaw} roughly corresponds to \code{@Initialized}, the
annotations are not aliased and you must use the ones that correspond to
the type-checker that you are running.
\end{sloppypar}


%% TODO: cut much of this, which is redundant with the text above.



An object is
\emph{raw} from the time that its constructor starts until its constructor
finishes.  This is relevant to the Nullness Checker because while the
constructor is executing --- that is, before initialization completes ---
a \<@NonNull>
field may be observed to be null, until that field is set.  In
particular, the Nullness Checker issues a warning for code like this:

\begin{Verbatim}
  public class MyClass {
    private @NonNull Object f;
    public MyClass(int x, int y) {
      // Error because constructor contains no assignment to this.f.
      // By the time the constructor exits, f must be initialized to a non-null value.
    }
    public MyClass(int x) {
      // Error because this.f is accessed before f is initialized.
      // At the beginning of the constructor's execution, accessing this.f
      // yields null, even though field f has a non-null type.
      this.f.toString();
    }
    public MyClass(int x, int y, int z) {
      m();
    }
    public void m() {
      // Error because this.f is accessed before f is initialized,
      // even though the access is not in a constructor.
      // When m is called from the constructor, accessing f yields null,
      // even though field f has a non-null type.
      this.f.toString();
    }
\end{Verbatim}

\noindent
In general, code can depend that field \<f> is not \<null>, because the
field is declared with a \refqualclass{checker/nullness/qual}{NonNull} type.
However, this guarantee does not hold for a partially-initialized object.

The Nullness Checker uses the \refqualclass{checker/nullness/qual}{Raw} annotation to indicate that an object
is not yet fully initialized --- that is, not all its \<@NonNull> fields have been
assigned.  Rawness is mostly relevant within the constructor, or for
references to \code{this} that escape the constructor (say, by being stored
in a field or passed to a method before initialization is complete).
Use of rawness annotations is rare in most code.

The most common use for the \<@Raw> annotation is for a helper routine that
is called by constructor.  For example:

\begin{Verbatim}
  class MyClass {
    Object field1;
    Object field2;
    Object field3;

    public MyClass(String arg1) {
      this.field1 = arg1;
      init_other_fields();
    }

    // A helper routine that initializes all the fields other than field1
    @EnsuresNonNull({"field2", "field3"})
    private void init_other_fields(@Raw MyClass this) {
      field2 = new Object();
      field3 = new Object();
    }
  }
\end{Verbatim}

For compatibility with Java 5/6/7, you can write the receiver
parameter in comments (see Section~\ref{receivers-and-imports-in-comments}):
\begin{Verbatim}
    private void init_other_fields(/*>>> @Raw MyClass this*/) {
\end{Verbatim}

% Most readers can
% skip this section on first reading; you can return to it once you have
% mastered the rest of the Nullness Checker.

\subsubsection{Rawness qualifiers\label{rawness-qualifiers}}

\begin{figure}
\includeimage{rawness}{3.5cm}
\caption{Partial type hierarchy for the Rawness Initialization type system.}
\label{fig-rawness-hierarchy}
\end{figure}

The rawness hierarchy is shown in Figure~\ref{fig-rawness-hierarchy}.
The rawness hierarchy contains these qualifiers:

\begin{description}

\item[\refqualclass{checker/nullness/qual}{Raw}]
  indicates a type that may contain a partially-initialized object.  In a
  partially-initialized object, fields that are annotated as
  \refqualclass{checker/nullness/qual}{NonNull} may be null because the field
  has not yet been assigned.  Within the constructor,
  \code{this} has \refqualclass{checker/nullness/qual}{Raw} type until all
  the \code{@NonNull} fields have been assigned.
  A partially-initialized object (\code{this} in a constructor) may be
  passed to a helper method or stored in a variable; if so, the method
  receiver, or the field, would have to be annotated as \<@Raw>.

% Cut this?
\item[\refqualclass{checker/nullness/qual}{NonRaw}]
  indicates a type that contains a fully-initialized object.  \code{NonRaw}
  is the default, so there is little need for a programmer to write this
  explicitly.

\item[\refqualclass{checker/nullness/qual}{PolyRaw}]
  indicates qualifier polymorphism over rawness (see
  Section~\ref{qualifier-polymorphism}).

\end{description}

% However, if the constructor makes
% a method call (passing \code{this} as a parameter or the receiver), then
% the called method could observe the object in an illegal state.

If a reference has
\code{@Raw} type, then all of its \code{@NonNull} fields are treated as
\refqualclass{checker/nullness/qual}{MonotonicNonNull}:  when read, they are
treated as being \refqualclass{checker/nullness/qual}{Nullable}, but when
written, they are treated as being
\refqualclass{checker/nullness/qual}{NonNull}.


The rawness hierarchy is orthogonal to the nullness hierarchy.  It
is legal for a reference to be \<@NonNull @Raw>, \<@Nullable @Raw>,
\<@NonNull @NonRaw>, or \<@Nullable @NonRaw>.  The nullness hierarchy tells
you about the reference itself:  might the reference be null?  The rawness
hierarchy tells you about the \<@NonNull> fields in the referred-to object:
might those fields be temporarily null in contravention of their
type annotation?
% It's a figure rather than appearing inline so as not to span page breaks
% in the printed manual.
Figure~\ref{fig-rawness-examples} contains some examples.

\begin{figure}
\begin{tabular}{l|l|l}
Declarations & Expression & Expression's nullness type, or checker error \\ \hline
\begin{minipage}{1.5in}
\begin{Verbatim}
class C {
  @NonNull Object f;
  @Nullable Object g;
  ...
}
\end{Verbatim}
\end{minipage} & & \\ \cline{2-3}
\<@NonNull @NonRaw C a;>
& \<a> & \<@NonNull> \\
& \<a.f> & \<@NonNull> \\
& \<a.g> & \<@Nullable> \\ \cline{2-3}
\<@NonNull @Raw C b;>
& \<b> & \<@NonNull> \\
& \<b.f> & \<@MonotonicNonNull> \\
& \<b.g> & \<@Nullable> \\ \cline{2-3}
\<@Nullable @NonRaw C c;>
& \<c> & \<@Nullable> \\
& \<c.f> & error: deref of nullable \\
& \<c.g> & error: deref of nullable \\ \cline{2-3}
\<@Nullable @Raw C d;>
& \<d> & \<@Nullable> \\
& \<d.f> & error: deref of nullable \\
& \<d.g> & error: deref of nullable \\
\end{tabular}
\caption{Examples of the interaction between nullness and rawness.
  Declarations are shown at the left for reference, but the focus of the
  table is the expressions and their nullness type or error.}
\label{fig-rawness-examples}
\end{figure}


% Does our implementation handle static fields soundly?  NO!  See issue
% #105.  Maybe document this?


\subsubsection{How an object becomes non-raw\label{becoming-non-raw}}

Within the constructor,
\code{this} starts out with \refqualclass{checker/nullness/qual}{Raw} type.
As soon as all of the \refqualclass{checker/nullness/qual}{NonNull} fields
in class $C$ have been initialized, then \code{this} is treated as
\refqualclass{checker/nullness/qual}{NonRaw}\code{(\emph{C})}.
This means that \code{this} is still being initialized, but all
initialization of $C$'s fields is complete, including all fields of supertypes.
Eventually, when all constructors complete, the type is simply
\refqualclass{checker/nullness/qual}{NonRaw}.

The Nullness Checker issues an error if the constructor fails to initialize
any \code{@NonNull} field.  This ensures that the object is in a legal (non-raw)
state by the time that the constructor exits.
This is different than Java's test for definite assignment (see
\href{https://docs.oracle.com/javase/specs/jls/se8/html/jls-16.html}{JLS ch.16}),
% , which requires that local
% variables (and blank \code{final} fields) must be assigned.  Java does not
% require that non-\code{final} fields be assigned, since
which does not apply to fields (except blank final ones, defined in
\href{https://docs.oracle.com/javase/specs/jls/se8/html/jls-4.html#jls-4.12.4}{JLS \S 4.12.4}) because fields
have a default value of null.


% and can only be passed to methods when the corresponding parameter is
% annotated with \refqualclass{checker/nullness/qual}{Raw}.  Similar
% restrictions apply to assigning \code{this} to a field.

All \code{@NonNull} fields must either have a
default in the field declaration, or be assigned in the constructor or in a
helper method that the constructor calls.  If
your code initializes (some) fields in a helper method, you will need to
annotate the helper method with an annotation such as
\refqualclass{checker/nullness/qual}{EnsuresNonNull}\code{(\{"field1", "field2"\})}
for all the fields that the helper method assigns.
It's a bit odd, but you use that same annotation, \code{@EnsuresNonNull},
to indicate that a primitive field has its value set in a helper method,
which is relevant when you supply the \code{-Alint=uninitialized}
command-line option (see Section~\ref{lint-uninitialized-section}).

% TODO:
% We need an
%   @EnsuresInitialized("b")
% that is analogous to
%   @EnsuresNonNull("b")
% when we are dealing with a field of primitive type.
% But, for now just use the same annotation, @EnsuresNonNull, for both purposes.


\subsubsection{Partial initialization\label{rawness-partial-initialization}}

So far, we have discussed rawness as if it is an all-or-nothing property:
an object is fully raw until initialization completes, and then it is no
longer raw.  The full truth is a bit more complex:  during the
initialization process, an object can be partially initialized, and as the
object's superclass constructors complete, its rawness changes.  The
Nullness Checker lets you express such properties when necessary.

Consider a simple example:

\begin{Verbatim}
class A {
  Object a;
  A() {
    a = new Object();
  }
}
class B extends A {
  Object b;
  B() {
    super();
    b = new Object();
  }
}
\end{Verbatim}

Consider what happens during execution of \<new B()>.

\begin{enumerate}
\item \<B>'s constructor begins to execute.  At this point, neither the
  fields of \<A> nor those of \<B> have been initialized yet.
\item \<B>'s constructor calls \<A>'s constructor, which begins to execute.
  No fields of \<A> nor of \<B> have been initialized yet.
\item \<A>'s constructor completes.  Now, all the fields of \<A> have been
  initialized, and their invariants (such as that field \<a> is non-null) can be
  depended on.  However, because \<B>'s constructor has not yet completed
  executing, the object being constructed is not yet fully initialized.
  When treated as an \<A> (e.g., if only the \<A> fields are accessed), the
  object is initialized (non-raw), but when treated as a \<B>, the object
  is still raw.
\item \<B>'s constructor completes.  The object is fully initialized
  (non-raw), if \<B>'s constructor was invoked via a \<new B()>
  expression.  On the other hand, if there was a \<class C extends B \{
  ... \}>, and \<B>'s constructor had been invoked from that, then the
  object currently under construction would \emph{not} be fully initialized
  --- it would only be initialized when treated as an \<A> or a \<B>, but
  not when treated as a \<C>.
\end{enumerate}

At any moment during initialization, the superclasses of a given class
can be divided into those that have completed initialization and those that
have not yet completed initialization.  More precisely, at any moment there
is a point in the class hierarchy such that all the classes above that
point are fully initialized, and all those below it are not yet
initialized.  As initialization proceeds, this dividing line between the
initialized and raw classes moves down the type hierarchy.

The Nullness Checker lets you indicate where the dividing line is between
the initialized and non-initialized classes.
You have two equivalent ways to indicate the dividing line:  \<@Raw>
indicates the first class \emph{below} the dividing line, or
\<@NonRaw(\emph{classliteral})> indicates the first class \emph{above} the
dividing line.

When you write \refqualclass{checker/nullness/qual}{Raw}\code{ MyClass x;}, that
means that variable \<x> is initialized for all superclasses of \<MyClass>,
and (possibly) uninitialized for \<MyClass> and all subclasses.

When you write \refqualclass{checker/nullness/qual}{NonRaw}\code{(Foo.class) MyClass
  x;}, that means that variable \<x> is initialized for \<Foo> and all its
superclasses, and (possibly) uninitialized for all subclasses of \<Foo>.

If \<A> is a direct superclass of \<B> (as in the example above), then
\<@Raw A x;> and \<@NonRaw(B.class) A x;> are equivalent declarations.
Neither one is the same as \<@NonRaw A x;>, which indicates that, whatever
the actual class of the object that \<x> refers to, that object is fully
initialized.  Since \<@NonRaw> (with no argument) is the default, you will
rarely see it written.

The example above lists 4 moments during construction.  At those moments,
the type of the object being constructed is:

\begin{enumerate}
\item
  \<@Raw Object>
\item
  \<@Raw Object>
\item
  \<@NonRaw(A.class) A>
  %% Not quite equivalent because the Java (non-qualified) type differs
  % ; equivalently, \<@Raw B>
\item
  \<@NonRaw(B.class) B>
\end{enumerate}

\paragraph{Example\label{rawness-partial-initialization-example}}

As another example, consider the following 12 declarations:

\begin{Verbatim}
    @Raw Object rO;
    @NonRaw(Object.class) Object nroO;
    Object o;

    @Raw A rA;
    @NonRaw(Object.class) A nroA;  // same as "@Raw A"
    @NonRaw(A.class) A nraA;
    A a;

    @NonRaw(Object.class) B nroB;
    @Raw B rB;
    @NonRaw(A.class) B nraB;  // same as "@Raw B"
    @NonRaw(B.class) B nrbB;
    B b;
\end{Verbatim}

In the following table, the type in cell C1 is a supertype of the type in
cell C2 if:  C1 is at least as high and at least as far left in the table
as C2 is.  For example, \<nraA>'s type is a supertype of those of \<rB>,
\<nraB>, \<nrbB>, \<a>, and \<b>.  (The empty cells on the top row are real
types, but are not expressible.  The other empty cells are not interesting
types.)

%BEGIN LATEX
\begin{smaller}
%END LATEX
\noindent
\begin{tabular}{|c|c|c|}

\hline
    \<@Raw Object rO;>
&
&
\\
\hline

    \<@NonRaw(Object.class) Object nroO;>
&
\begin{minipage}{2in}
\begin{Verbatim}
@Raw A rA;
@NonRaw(Object.class) A nroA;
\end{Verbatim}
\end{minipage}
&
    \<@NonRaw(Object.class) B nroB;>
\\
\hline

&
    \<@NonRaw(A.class) A nraA;>
&
\begin{minipage}{1.75in}
\begin{Verbatim}
@Raw B rB;
@NonRaw(A.class) B nraB;
\end{Verbatim}
\end{minipage}
\\
\hline

&
&
    \<@NonRaw(B.class) B nrbB;>
\\
\hline

    \<Object o;>
&
    \<A a;>
&
    \<B b;>
\\
\hline
\end{tabular}
%BEGIN LATEX
\end{smaller}
%END LATEX



% \urldef{\jlsconstructorbodyurl}{\url}{https://docs.oracle.com/javase/specs/jls/se8/html/jls-8.html#jls-8.8.7}
% (Recall that the superclass constructor is called on the first line, or is
% inserted automatically by the compiler before the first line, see
% \href{\jlsconstructorbodyurl}{JLS \S8.8.7}.)



\subsubsection{More details about rawness checking\label{rawness-checking}}


\paragraph{Suppressing warnings\label{rawness-checking-suppressing-warnings}}

\begin{sloppypar}
You can suppress warnings related to partially-initialized objects with
\<@SuppressWarnings("rawness")>.  Do not confuse this with the unrelated
\<@SuppressWarnings("rawtypes")> annotation for non-instantiated generic types!
\end{sloppypar}


\paragraph{Checking initialization of all fields, not just \code{@NonNull} ones\label{rawness-checking-checking-all-fields}}

When the \code{-Alint=uninitialized} command-line option is provided, then
an object is considered raw until \emph{all} its fields are assigned, not
just the \code{@NonNull} ones.  See Section~\ref{lint-uninitialized-section}.


\paragraph{Use of method annotations\label{rawness-checking-method-annotations}}

A method with a raw receiver often assumes that a few fields (but not all
of them) are non-null, and sometimes sets some more fields to non-null
values.  To express these concepts, use the
\refqualclass{checker/nullness/qual}{RequiresNonNull},
\refqualclass{checker/nullness/qual}{EnsuresNonNull}, and
\refqualclass{checker/nullness/qual}{EnsuresNonNullIf} method annotations;
see Section~\ref{nullness-method-annotations}.


% Should we change the terminology?
\paragraph{The terminology ``raw''\label{rawness-checking-terminology-raw}}

The name ``raw'' comes from a research paper that proposed this
approach~\cite{FahndrichL2003}.
A better name might have been ``not yet initialized'' or ``partially
initialized'', but the term ``raw'' is now well-known.
The \refqualclass{checker/nullness/qual}{Raw}
annotation has nothing to do with the raw types of Java Generics.


% LocalWords:  NonNull plugin quals un NonNullExampleWithWarnings java ahndrich
% LocalWords:  NotNull IntelliJ FindBugs Nullable TODO Alint nullable NNEL JSR
% LocalWords:  DefaultLocation Nullness PolyNull nullness AnnotateNullable JLS
% LocalWords:  Daikon JastAdd javac DefaultQualifier boolean MyEnumType NonRaw
% LocalWords:  NullnessAnnotatedTypeFactory NullnessVisitor MonotonicNonNull
% LocalWords:  inferencer Nonnull CheckForNull UnknownNullness rawtypes de ch
% LocalWords:  castNonNull NullnessUtil assertNotNull codebases checkNotNull
% LocalWords:  Nullability typeargs nulltest EnsuresNonNullIf listFiles faq
% LocalWords:  isDirectory AssertionError intraprocedurally SuppressWarnings rB
% LocalWords:  FindBugs's getObject RequiresNonNull EnsuresNonNull KeyFor
% LocalWords:  nonnull EnsuresNonNull arg orElse isPresent bField qual ccc
% LocalWords:  keySet getField keyfor param TypeName containsValue indexOf nraA
% LocalWords:  lastIndexOf deref getProperty getProperties classliteral MyClass
% LocalWords:  typeofthis nraB nrbB rO nroO nroB containsKey enum
% LocalWords:  JUnit's field1 field2 superclasses Foo C1 C2 PolyRaw
% LocalWords:  NullnessChecker redundantNullComparison instanceof
% LocalWords:  noInitForMonotonicNonNull UnknownInitialization isArray
% LocalWords:  UnderInitialization getComponentType AsuppressWarnings
% LocalWords:  isEmpty AssumeAssertion cleanroom varargs OtherClass
% LocalWords:  NullnessRawnessChecker mymap mykey KeyType ValueType
% LocalWords:  dereferenced dereference Dereferencing initializers
% LocalWords:  initHelper helperMethod NoSuchElementException classpath
% LocalWords:  forbidnonnullarraycomponents lombok
% LocalWords:  NotOnlyInitialized
