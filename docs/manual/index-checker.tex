%\htmlhr
\chapter{Index Checker for array bounds\label{index-checker}}

The Index Checker warns about any potentially out-of-bounds accesses to sequence-like
data structures, such as arrays, lists, and strings. A sequence-like data structure
has a defined length and is accessed with an index that starts with zero. The Index Checker maintains
information about both sequence-like data structures and potential indices.

The Index Checker warns about several possible errors:

\begin{enumerate}
\item Accessing a sequence-like data structure with an index that
  might be negative. The Index Checker will warn that the expression
  used as the index might be too low.
\item Accessing a sequence-like data structure with an index
  that might be larger than the data structure's length. The Index
  Checker will warn that the expression used as the index might be
  too high.
\item Intializing an array with a possibly negative length. The Index
  Checker will warn that the length expression might be too low.
\item Assigning a possibly out-of-bounds value into a declared
  index type. The Index Checker allows programmers to annotate
  that particular variables are safe to use as accesses. If an
  assignment (or passed parameter, or return value) might violate
  the assertion made by the annotation, the Index Checker will
  issue a type error.

  \end{enumerate}

Typically, programmers need to write only a few annotations to use the
Index Checker. The Index Checker will infer properties of indexes from
the code around them. For example, it will infer that \<x> is positive
within the then block of an \code{if (x > 0)} statement. However, the
Index Checker does not infer method pre-conditions or post-conditions,
which the programmer needs to write as annotations. For instance,
if an integer that is passed into a method is guaranteed to be a safe
index for the array ''myArray'', the programmer might need to
write an @IndexFor(''myArray'') annotation.

The Index Checker is a composition of several checkers: the Lower
Bound Checker (section~\ref{index-lowerbound}), which checks whether
potential accesses may be too low; the Upper Bound Checker
(section~\ref{index-upperbound}), which checks whether potential
accesses may be too high; the MinLen Checker
(section~\ref{index-minlen}), which estimates the minimum length of
each sequence; and the SameLen Checker (section~\ref{index-samelen}),
which finds arrays of equal length. All of the qualifiers of these
subcheckers are considered ''index'' annotation (and warnings from any
of them can be suppressed with @SuppressWarnings(''index''). In
addition, the Index Checker maintains some qualifiers of its own, each
of which is the combination of at least two qualifiers from different
index subcheckers.

The Index annotations are:
\begin{description}
\item[\refqualclasswithparams{checker/index/qual}{IndexFor}{String[] names}]
  The value is a valid index for the named arrays. This type is the combination
  of the Lower Bound Checker's
  \refqualclass{checker/lowerbound/qual}{NonNegative} and the Upper Bound
  Checker's \refqualclasswithparams{checker/upperbound/qual}{LTLengthOf}{String[] names}.
 \item[\refqualclasswithparams{checker/index/qual}{IndexOrHigh}{String[] names}]
   The value is either a valid index for each named array; or equal to the
   length of one or more of the named arrays, and a valid index for the
   others. This type is the combination of the Lower Bound Checker's
  \refqualclass{checker/lowerbound/qual}{NonNegative} and the Upper Bound
  Checker's
  \refqualclasswithparams{checker/upperbound/qual}{LTEqLengthOf}{String[] names}.
\end{description}

\section{Running the Index Checker\label{index-running}}

To run the Index Checker, run the command

\begin{alltt}
  javac -processor index \emph{MyJavaFile}.java
\end{alltt}

\section{The Lower Bound Checker\label{index-lowerbound}}

The Lower Bound Checker warns about indices that might be
too low to access a zero-indexed sequence. It issues an error when
a sequence-like data structure is indexed by an integer that is not
\refqualclass{checker/lowerbound/qual}{Positive} or \refqualclass{checker/lowerbound/qual}{NonNegative},
or when an array is created with a length expression that is not
\refqualclass{checker/lowerbound/qual}{Positive} or \refqualclass{checker/lowerbound/qual}{NonNegative}.
The
Lower Bound Checker's type system uses the following qualifiers:

\begin{description}
\item[\refqualclass{checker/lowerbound/qual}{Positive}]
  The value is 1 or greater, so it is not too low to be used as an index.
\item[\refqualclass{checker/lowerbound/qual}{NonNegative}]
  The value is 0 or greater, so it is not too low to be used as an index.
\item[\refqualclass{checker/lowerbound/qual}{GTENegativeOne}]
  The value is -1 or greater.
  It may not be used as an index for a sequence, because it might be too low.
\item[\refqualclass{checker/lowerbound/qual}{LowerBoundUnknown}]
  There is no information about the value.
  It may not be used as an index for a sequence, because it might be too low.
\end{description}

\begin{figure}
  \includeimage{lowerbound}{7cm}
  \caption{The type hierarchy used by the Lower Bound Checker.}
  \label{fig-lowerbound-types}
\end{figure}

Typically, a programmer only needs to write these annotations on
fields and method signatures, because the Lower Bound Checker infers
qualifiers within method bodies.
In general, the Lower Bound Checker attempts to infer as much information
as possible from the source code.
Detailed documentation about the inferences performed can be found
in the Javadoc comments in its source code.


\section{The Upper Bound Checker\label{index-upperbound}}

The Upper Bound Checker warns about accesses to sequences with indexes that might be
too high. To do this, it maintains information about which potential indices are safe
for which sequence-like data structures. It issues an error when a sequence \code{arr}
is indexed by an integer that is not of type \code{@LTLengthOf({''arr''})}
or \code{@LTOMLengthOf({''arr''})}.
The Upper Bound Checker's type system uses the following qualifiers:

\begin{description}
\item[\refqualclasswithparams{checker/upperbound/qual}{LTLengthOf}{String[] names}]
  An expression with this type
  has value less than the length of each sequence listed in \<names>.
  The expression may be used as an index into any of those sequences.
  \code{@LTLengthOf({"a", "b"})} is a subtype of both
  \code{@LTLengthOf("a")} and \code{@LTLengthOf("b")}.
\item[\refqualclasswithparams{checker/upperbound/qual}{LTEqLengthOf}{String[] names}]
  An expression with this type
  has value less than or equal to the length of each sequence listed in \<names>.
  It may not be used as an index for a sequence, because it might be too high.
  \code{@LTEqLengthOf({"a", "b"})} is a subtype of both
  \code{@LTEqLengthOf("a")} and \code{@LTEqLengthOf("b")}.
\item[\refqualclasswithparams{checker/upperbound/qual}{LTOMLengthOf}{String[] names}]
  An expression with this type
  has value less than one less than the length of each sequence listed in \<names>. This type exists to allow the checker to infer the safety of loops of
  the form:
\begin{Verbatim}
  for (int i = 0; i < array.length - 1; ++i) {
    arr[i] = arr[i+1];
  }
\end{Verbatim}
It may always used as an index for a sequence listed in \<names>, but
should rarely (if ever) be written by the programmer; usually
\refqualclasswithparams{checker/upperbound/qual}{LTLengthOf}{String[] names}
should be written instead.
  \code{@LTOMLengthOf({"a", "b"})} is a subtype of both
  \code{@LTOMLengthOf("a")} and \code{@LTOMLengthOf("b")}.
\item[\refqualclass{checker/upperbound/qual}{UpperBoundUnknown}]
  There is no information about the value of an expression with this type.
  It may not be used as an index for a sequence, because it might be too high.
  This type is the top type, and should never need to be written by the
  programmer.
\item[\refqualclass{checker/upperbound/qual}{UpperBoundBottom}]
  This is the bottom type for the Upper Bound type system. It should
  never need to be written by the programmer.
  \end{description}

\begin{figure}
  \includeimage{upperbound}{7cm}
  \caption{The type hierarchy used by the Upper Bound Checker.}
  \label{fig-upperbound-types}
\end{figure}

Typically, a programmer only needs to write these annotations on
fields and method signatures, because the Upper Bound Checker infers
qualifiers within method bodies.

\section{The MinLen Checker\label{index-minlen}}

The MinLen Checker estimates, for each sequence expression, how long its value
might be at run time.  The MinLen Checker computes a minimum length that
the sequence is guaranteed to have.

The main reason for the MinLen Checker is to handle fixed-length arrays
that are indexed by a compile-time constant.
There are two different situations when \<arr[i]> is legal because \<i> is
not too large:
\begin{itemize}
\item
  when \<i> is known to be less than the length of \<arr> independent of the length of
  \<arr>.  This fact is represented by giving \<i> the type \refqualclasswithparams{checker/upperbound/qual}{LTLengthOf}{"arr"}.
\item
  when \<i> is a compile-time constant but the length of \<arr> is known to
  be greater.  For example, consider the following code:
\end{itemize}

\begin{Verbatim}
  String getSecondElement(String[] arr) {
    return arr[1];
  }
\end{Verbatim}
  This is legal if \<arr> has at least two elements, which can be indicated
  in this way:
\begin{Verbatim}
  String getSecondElement(String @MinLen(2) [] arr) {
    return arr[1];
  }
\end{Verbatim}

The MinLen Checker uses the following type hierarchy:

\begin{description}
\item[\refqualclasswithparams{checker/minlen/qual}{MinLen}{int value}]
  An expression with this type represents a sequential structure
  with at least \code{value} elements.  The default annotation is
  \code{@MinLen(0)}, since an arbitrary object has at least 0 elements.
  In general, \code{@MinLen(x)} is a subtype of \code{@MinLen(x-1)}.
\item[\refqualclass{checker/minlen/qual}{MinLenBottom}]
  This is the bottom type for the MinLen type system.
  Programmers should rarely need to write it.
  \code{null} has this type.
  \end{description}

\begin{figure}
  \includeimage{minlen}{7cm}
  \caption{The type hierarchy used by the MinLen Checker.}
  \label{fig-minlen-types}
\end{figure}

\section{The SameLen Checker\label{index-samelen}}

The SameLen Checker determines whether sequences have the same length.

The main reason for the SameLen Checker is to handle cases where two
arrays are known to be the same length, and one is accessed using a
valid index for the other.

The SameLen Checker uses the following type hierarchy:

\begin{description}
  \item[\refqualclass{checker/samelen/qual}{SameLenUnknown}]
  This is the top type for the SameLen type system.
  Programmers should never need to write it, since it
  is automatically inferred for any expression that does
  not have a more specific type.
\item[\refqualclasswithparams{checker/samelen/qual}{SameLen}{String[] names}]
  An expression with this type represents a sequence that has the
  same length as the other sequences named in \<names>. In general,
  \code{@SameLen} types that have non-intersecting lists of names
  are \textit{not} subtypes of each other. However, if at least one
  sequence is named by both types, the types are actually the same,
  because all the named sequences must have the same length.
\item[\refqualclass{checker/samelen/qual}{SameLenBottom}]
  This is the bottom type for the SameLen type system.
  Programmers should rarely need to write it.
  \code{null} has this type.
  \end{description}

\begin{figure}
  \includeimage{samelen}{7cm}
  \caption{The type hierarchy used by the SameLen Checker.}
  \label{fig-samelen-types}
\end{figure}
