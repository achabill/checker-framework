%\htmlhr

% Reinstate when lists are supported:
% \chapter{Index Checker for sequence bounds (arrays, strings, lists)\label{index-checker}}
\chapter{Index Checker for sequence bounds (arrays and strings)\label{index-checker}}

The Index Checker warns about potentially out-of-bounds accesses to sequence
data structures, such as arrays
% , lists,
and strings.
>>>>>>> 15c85eab6a95994155e8a8e17149b072553bc1f6

The Index Checker prevents \<IndexOutOfBoundsException>s that result from
an index expression that might be negative or might be equal to or larger
than the sequence's length.
It also prevents \<NegativeArraySizeException>s that result from a negative
array dimension in an array creation expression.
(A caveat:  the Index Checker does not check for arithmetic overflow.  If
an expression overflows, the Index Checker might fail to warn about a
possible exception.)

% Here's a pathological example of overflow leading to unsoundness:
%
% public class IndexOverflow {
%     public static void main(String[] args) {
%         @Positive int x = 1073741825; // 2 ^ 30 + 1
%         @Positive int x2 = x + x; // 2 ^ 31 + 2 == - 2 ^ 31 + 2
%         int[] a = new int[0];
%         if (x2 < a.length) {
%             a[x2] = 42;
%         }
%     }
% }

The programmer can write annotations that indicate which expressions are
indices for which arrays.  The Index Checker prohibits any operation that
may violate these properties, and the Index Checker takes advantage of
these properties when verifying indexing operations.
%
Typically, a programmer writes few annotations, because the Index Checker
infers properties of indexes from
the code around them. For example, it will infer that \<x> is positive
within the \<then> block of an \code{if (x > 0)} statement.
The programmer does need to write field types and method pre-conditions or post-conditions. For instance,
if a method's formal parameter is used as an index for
\<myArray>, the programmer might need to
write an \refqualclasswithparams{checker/index/qual}{IndexFor}{myArray}
annotation on the formal parameter's types.

To run the Index Checker, run the command

\begin{alltt}
  javac -processor index \emph{MyJavaFile}.java
\end{alltt}


\section{Index Checker structure and annotations\label{index-annotations}}

The main annotations used by the Index Checker are these:

\begin{description}
\item[\refqualclasswithparams{checker/index/qual}{IndexFor}{String[] names}]
  The value is a valid index for the named arrays.  For example, the
  \sunjavadoc{java/lang/String.html\#charAt-int-}{String.charAt(int)}
  method is declared as

  \begin{Verbatim}
  class String {
    char charAt(@IndexFor("this") index) { ... }
  }
  \end{Verbatim}

  More generally, a variable
  declared as \<@IndexFor("someArray") int i> has type
  \<@IndexFor("someArray") int> and its run-time value is guaranteed to be
  non-negative and less than the length of \<someArray>.  You could also
  express this as
  \<\refqualclass{checker/index/qual}{NonNegative}
  \refqualclasswithparams{checker/index/qual}{LTLengthOf}{"someArray"}
  int i>,
  but \<@IndexFor("someArray") int i> is more concise.

 \item[\refqualclasswithparams{checker/index/qual}{IndexOrHigh}{String[] names}]
   The value is non-negative and is less than or equal to the length of
   each named array.  This type combines
  \refqualclass{checker/index/qual}{NonNegative} and
  \refqualclass{checker/index/qual}{LTEqLengthOf}.

  For example, the
  \sunjavadoc{java/util/Arrays.html\#binarySearch-java.lang.Object:A-int-int-java.lang.Object-}{Arrays.binarySearch}
   method is declared as

  \begin{mysmall}
  \begin{Verbatim}
  class Arrays {
    int binarySearch(Object[] a, @IndexFor("#1") int fromIndex, @IndexOrHigh("#1") int toIndex, Object key)
  }
  \end{Verbatim}
  \end{mysmall}

 \item[\refqualclasswithparams{checker/index/qual}{IndexOrLow}{String[] names}]
   The value is -1 or is a valid index for
   each named array.  This type combines
  \refqualclass{checker/index/qual}{GTENegativeOne} and
  \refqualclass{checker/index/qual}{LTEqLengthOf}.

  For example, the
  \sunjavadoc{java/lang/String.html\#indexOf-java.lang.String-}{String.indexOf(String)}
  method is declared as

  \begin{Verbatim}
  class String {
    @IndexOrLow("this") int indexOf(String str) { ... }
  }
  \end{Verbatim}

\end{description}

Internally, the Index Checker computes several types of information about
program variables, recording each one in a different type system.  The
Index Checker does type-checking of all these type systems at once; the
type systems cooperate to permit precise verification of sequence indexing
operations.  This manual discusses each type system in a different
section:
\begin{itemize}
\item
  the lower bound on an integer, such as whether it is known to be positive
  (Section~\ref{index-lowerbound})
\item
  the upper bound an an integer, such as whether it is less than the length
  of a given array (Section~\ref{index-upperbound})
\item
  the minimum length of a sequence, such ``\<myArray> contains at least 3
  elements'' (Section~\ref{index-minlen})
\item
  whether two sequences have the same length (Section~\ref{index-samelen})
\end{itemize}


\section{Lower bounds\label{index-lowerbound}}

The Index Checker issues an error when
a sequence is indexed by an integer that might be negative.
It uses a type system (Figure~\ref{fig-index-int-types}) with the following
qualifiers:

\begin{description}
\item[\refqualclass{checker/index/qual}{Positive}]
  The value is 1 or greater, so it is not too low to be used as an index.
\item[\refqualclass{checker/index/qual}{NonNegative}]
  The value is 0 or greater, so it is not too low to be used as an index.
\item[\refqualclass{checker/index/qual}{GTENegativeOne}]
  The value is -1 or greater.
  It may not be used as an index for a sequence, because it might be too low.
  (``\<GTE>'' stands for ``Greater Than or Equal to''.)
\item[\refqualclass{checker/index/qual}{LowerBoundUnknown}]
  There is no information about the value.
  It may not be used as an index for a sequence, because it might be too low.
\end{description}

\begin{figure}
\begin{center}
  \hfill
  \includeimagenocentering{lowerbound}{5cm}
  ~~~~\hfill~~~~
  \includeimagenocentering{upperbound}{7cm}
  \hfill
\end{center}
  \caption{The two type hierarchies for integer types used by the Index
    Checker.  On the left is a type system for lower bounds.  On the right
    is a type system for upper bounds.  Qualifiers written in gray should
    never be written in source code; they are used internally by the type
    system. ''myArray'' in the Upper Bound qualifiers is an example of
    an array's name.}
  \label{fig-index-int-types}
\end{figure}


\section{Upper bounds\label{index-upperbound}}

The Index Checker issues an error when a sequence index might be
too high. To do this, it maintains information about which expressions are
safe indices for which sequences.
It uses a type system (Figure~\ref{fig-index-int-types}) with the following
qualifiers:

It issues an error when a sequence \code{arr}
is indexed by an integer that is not of type \code{@LTLengthOf({''arr''})}
or \code{@LTOMLengthOf({''arr''})}.

\begin{description}

\item[\refqualclasswithparams{checker/index/qual}{LTLengthOf}{String[] names}]
  An expression with this type
  has value less than the length of each sequence listed in \<names>.
  The expression may be used as an index into any of those sequences,
  if it is non-negative.
  For example, an expression of type \code{@LTLengthOf("a") int} might be
  used as an index to \<a>.
  The type \code{@LTLengthOf({"a", "b"})} is a subtype of both
  \code{@LTLengthOf("a")} and \code{@LTLengthOf("b")}.
  (``\<LT>'' stands for ``Less Than''.)

\item[\refqualclasswithparams{checker/index/qual}{LTEqLengthOf}{String[] names}]
  An expression with this type
  has value less than or equal to the length of each sequence listed in \<names>.
  It may not be used as an index for these sequences, because it might be too high.
  \code{@LTEqLengthOf({"a", "b"})} is a subtype of both
  \code{@LTEqLengthOf("a")} and \code{@LTEqLengthOf("b")}.
  (``\<LTEq>'' stands for ``Less Than or Equal to''.)

\item[\refqualclasswithparams{checker/index/qual}{LTOMLengthOf}{String[] names}]
  An expression with this type
  has value at least 2 less than the length of each sequence listed in \<names>.
  It may always used as an index for a sequence listed in \<names>, if it is
  non-negative.

  This type exists to allow the checker to infer the safety of loops of
  the form:
\begin{Verbatim}
  for (int i = 0; i < array.length - 1; ++i) {
    arr[i] = arr[i+1];
  }
\end{Verbatim}

  This annotation should rarely (if ever) be written by the programmer; usually
  \refqualclasswithparams{checker/index/qual}{LTLengthOf}{String[] names}
  should be written instead.
  \code{@LTOMLengthOf({"a", "b"})} is a subtype of both
  \code{@LTOMLengthOf("a")} and \code{@LTOMLengthOf("b")}.
  (``\<LTOM>'' stands for ``Less Than One Minus'', because another way of
  saying ``at least 2 less than \<a.length>'' is ``less than \<a.length-1>''.)

\item[\refqualclass{checker/index/qual}{UpperBoundUnknown}]
  There is no information about the upper bound on the value of an expression with this type.
  It may not be used as an index for a sequence, because it might be too high.
  This type is the top type, and should never need to be written by the
  programmer.

\item[\refqualclass{checker/index/qual}{UpperBoundBottom}]
  This is the bottom type for the upper bound type system. It should
  never need to be written by the programmer.

\end{description}


\section{Array minimum lengths\label{index-minlen}}

The Index Checker estimates, for each sequence expression, how long its value
might be at run time.  The Index Checker computes a minimum length that
the sequence is guaranteed to have.  This enables the Index Checker to
verify indices that are compile-time constants.  For example, this code:

\begin{Verbatim}
  String getThirdElement(String[] arr) {
    return arr[2];
  }
\end{Verbatim}

\noindent
is legal if \<arr> has at least three elements, which can be indicated
in this way:

\begin{Verbatim}
  String getThirdElement(String @MinLen(3) [] arr) {
    return arr[2];
  }
\end{Verbatim}

When the index is not a compile-time constant, as in \<arr[i]>, then the
Index Checker depends not on a \<@MinLen> annotation but on \<i> being
annotated as
\refqualclasswithparams{checker/index/qual}{LTLengthOf}{"arr"}.

Array minimum lengths use the following type qualifiers
(Figure~\ref{fig-index-array-types}):

\begin{description}
\item[\refqualclasswithparams{checker/index/qual}{MinLen}{int value}]
  The value of an expression of this type is a sequence with at least
  \code{value} elements.  The default annotation is
  \code{@MinLen(0)}, and it may be applied to non-sequences.
  \code{@MinLen($x$)} is a subtype of \code{@MinLen($x-1$)}.
\item[\refqualclass{checker/index/qual}{MinLenBottom}]
  This is the bottom type for the MinLen type system.
  Programmers should rarely need to write it.
  \code{null} has this type.
  \end{description}

\begin{figure}
\begin{center}
  \hfill
  \includeimagenocentering{minlen}{7cm}
  ~~~~\hfill~~~~
  \includeimagenocentering{samelen}{3.5cm}
  \hfill
\end{center}
  \caption{The two type hierarchies for array types used by the Index
    Checker.  On the left is a type system for minimum array lengths.  On
    the right is a type system for arrays of equal length.  Qualifiers
    written in gray should never be written in source code; they are used
    internally by the type system.}
  \label{fig-index-array-types}
\end{figure}

\section{Arrays of the same length\label{index-samelen}}

The Index Checker determines whether two or more sequences have the same length.
This enables it to verify that all the indexing operations are safe in code
like the following:

\begin{Verbatim}
  boolean lessThan(double[] arr1, double @SameLen("#1") [] arr2) {
    for (int i=0; i<arr1.length; i++) {
      if (arr1[i] < arr2[i]) {
        return true;
      } else if (arr1[i] > arr2[i]) {
        return false;
      }
    }
    return false;
  }
\end{Verbatim}

When needed, you can specify which sequences have the same length using the following type qualifiers (Figure~\ref{fig-index-array-types}):

\begin{description}
\item[\refqualclasswithparams{checker/index/qual}{SameLen}{String[] names}]
  An expression with this type represents a sequence that has the
  same length as the other sequences named in \<names>. In general,
  \code{@SameLen} types that have non-intersecting sets of names
  are \textit{not} subtypes of each other. However, if at least one
  sequence is named by both types, the types are actually the same,
  because all the named sequences must have the same length.
\item[\refqualclass{checker/index/qual}{SameLenUnknown}]
  No information is known about which other sequences have the same length
  as this one.
  This is the top type, and programmers should never need to write it.
\item[\refqualclass{checker/index/qual}{SameLenBottom}]
  This is the bottom type, and programmers should rarely need to write it.
  \code{null} has this type.
  \end{description}

%%  LocalWords:  NegativeArraySizeException pre myArray IndexFor someArray
%%  LocalWords:  MyJavaFile LTLengthOf LTEqLengthOf GTENegativeOne GTE
%%  LocalWords:  LowerBoundUnknown LTOMLengthOf LTEq LTOM UpperBoundBottom
%%  LocalWords:  UpperBoundUnknown MinLen MinLenBottom SameLen
%%  LocalWords:  SameLenUnknown SameLenBottom
